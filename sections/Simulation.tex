\section{Simulation}
\subsection{Switches}
Switches allow for the selective connection or disconnection of different loads while power is applied to the circuit. When it comes to resistive loads, they exhibit no dynamic response to changes in voltage or current. This implies that the ideal resistor model \eqref{eq:ideal_resistor} holds true.
\leavevmode\newline

In contrast, when dealing with \textit{inductive loads}, it is important that the current through the inductor ($I_L$) is zero at the time of connection for the derived equations to hold true. This is due to the property of inductors resisting changes in current. \citep{inductor}

\subsubsection{Considerations when Disconnecting Inductive Loads}
However, disconnecting an inductive load presents a different scenario. Here, an ideal switch is presumed, which disconnects instantaneously as $\Delta t$ tends to zero. Nevertheless, due to the inductive property of resisting changes in current, the current through the inductor will be non-zero at the instant of switching. This leads to an instantaneous reduction of current to zero, thus causing a voltage spike across the inductor according to Faraday's law: \citep{inductor2}

\begin{equation}
V_L = -L \frac{di}{dt}
\end{equation}

where $V_L$ is the voltage across the inductor, $L$ is the inductance, and $\frac{di}{dt}$ is the rate of change of current.

\subsubsection{Modelling Voltage Spike on Inductive Load Disconnection}

This spike, mathematically, approaches negative infinity, which is not physically meaningful. In reality, the voltage will increase abruptly, but the breakdown of the switch will eventually occur, leading to a non-instantaneous reduction of current. To model this scenario, the voltage across the diode in the cycle following the switch-off event is estimated as:

\begin{equation}
V_{L_{n+1}} = - \frac{L \cdot i_{n}}{h}
\end{equation}

where $V_{L_{n+1}}$ is the voltage across the inductor at the next cycle, $i_n$ is the current through the inductor at the current cycle, and $h$ is the time-step size.

\subsubsection{Post Disconnection of Inductive Load}
In the subsequent cycles after disconnection, the voltage and current across the inductor are both equal to zero. The voltage across the resistor will also be zero after the switch is flipped, since it is not a reactive element.

\pagebreak
\subsection{Algorithmic Implementation}
The numerical analysis of the circuit is implemented in MATLAB using the \texttt{powerSupply} function. This function applies the fourth-order Runge-Kutta method (RK4) to solve the governing differential equations.
\begin{itemize}
\item To manage varying load conditions, the design incorporates polymorphism by utilising four different sub-functions. Each sub-function handles the characteristic equations specific to a particular load condition.
\item RK4 is applied in each scenario to compute the state of the circuit at the next time step, based on the current state and input voltages (.i.e. determine the characteristic voltages and currents of the circuit)
\item The primary \small{\texttt{powerSupply}} function parses the \small{\texttt{mode}} argument to determine the current load condition. It then dispatches the calculation to the appropriate sub-function. This is managed through a switch-case control flow mechanism.
\item After obtaining the next state from the appropriate sub-function, \small{\texttt{powerSupply}} calculates the remaining circuit parameters, $i_c(t), i_z(t), i_s, i_d(t)$.
\end{itemize}
The code is organised modularly, separating the overall simulation flow from the specific equations for different load conditions. This structure makes the design robust, readable, maintainable, and easily extendable for additional load conditions or complexities.

\lstinputlisting[caption=Primary MATLAB function for simulating the power supply operations using RK4]{code/powerSupply.m}

\subsubsection{No Load Scenario}
\lstinputlisting[caption=Sub-function for the no-load scenario of the power supply simulation]{code/powerSupply_noLoad.m}

\subsubsection{Resistive Load Scenario}
\lstinputlisting[caption=Sub-function for the resistive load scenario of the power supply simulation]{code/powerSupply_resistiveLoad.m}

\subsubsection{Inductive Load Scenario}
\lstinputlisting[caption=Sub-function for the inductive load scenario of the power supply simulation]{code/powerSupply_inductiveLoad.m}

\subsubsection{Full Load Scenario}
\lstinputlisting[caption=Sub-function for the full-load scenario of the power supply simulation]{code/powerSupply_fullLoad.m}

\pagebreak
\subsection{Timestep Selection}
The time-step $h$ used in the RK4 method was determined using,
\begin{equation}
	h = \frac{T}{c}
\end{equation}
where $T$ is the period of the input sinusoid, or $0.02$  (corresponding to a 50 Hz signal), and $c$ is the desired number of samples per period (also known as the sampling rate).
\begin{itemize}
	\item A sensible choice for \(h\) would be sufficiently small to accurately capture the behavior of the system within each period of the sinusoidal input. Applying the Nyquist sampling theorem, \(c\) should be at least twice the highest frequency present in the signal to avoid aliasing. Therefore, a minimum value of \(c = 2 \times 50 = 100\) samples per second would be needed.
	\item However, for numerical solutions of differential equations using the RK4 method, we may require a higher sampling rate to capture the system dynamics more accurately. In this case, \(c\) could be set to a much larger value, leading to a smaller time step \(h\).
	\item For instance, setting \(c = 1000\) samples per second would result in a time step $h = \frac{0.02}{1000} = 20\mu s$, giving us a high-resolution representation of the system's behaviour within each period of the sinusoidal input.
\end{itemize}

A step size of $80\mu s$, $40 \mu s$ and $20 \mu s$, corresponding to a sampling rate of $250$, $500$ and $1000$ cycles per second (respectively) produced sensible results in the simulation across all loads. $40 \mu s$ was chosen as the final value which offered a reasonable compromise between computation time and quality of results. 

\pagebreak
\subsection{Results}
The results of the simulation with a time step of $h = 40 \mu s$ are shown across all four load conditions. Each respective plot shows the voltages at node \textcircled{2} and \textcircled{3} alongside the inductor $i_l(t)$ and zener $i_z(t)$ currents, with respect to time.

\subsubsection{No Load}
\begin{figure}[H]
    \centering
    \includegraphics[width=\textwidth]{/graphics/exports/psp_NoLoad_h_4e-05.png}
    \caption{Node voltages and currents under No Load conditions with $h=4\mu s$}
\end{figure}
\begin{itemize}
	\item The first plot shows the voltages $e_2$ and $e_3$ under No load conditions with a timestep (h) of $4x10^{-05}$ seconds. The Voltage $e_2$ levels off at around 13.7V and the Voltage $e_3$ levels off at around 6.1V. The leveling off of voltages $e_2$ and $e_3$ is due to the steady-state reached in the power supply circuit. The voltage $e_2$ levels off at around 13.7V, indicating the equilibrium reached in the capacitor's charging and discharging processes. Likewise,The voltage $e_3$ levels off at around 6.1V, suggesting that the zener diode is biased in the reverse direction and has reached a steady-state voltage.
    \item The second plot displays the currents $I_z$ and $I_L$ under the resistive load conditions. Current $I_L$ represents the inductor current. It initially increases due to changing voltage across the inductor but eventually reaches a steady-state level. The back EMF balances the applied voltage, resulting in a constant current flow. In the plot, $I_z$ levels off at approximately 0.025A, indicating the equilibrium in the inductor's behavior. Current $I_z$ represents the zener diode current. It remains at 0A throughout the simulation because the zener diode is in reverse bias when the voltage across it is below the zener voltage ($V_z$). The zener diode blocks significant current flow, acting as an open circuit. The voltage $e_3$ remains below $V_z$, maintaining the zener diode in a non-conducting state.
\end{itemize}

\subsubsection{Inductive Load}
\begin{figure}[H]
    \centering
    \includegraphics[width=\textwidth]{/graphics/exports/psp_InductiveLoad_h_4e-05.png}
    \caption{Node voltages and currents under Inductive Load conditions with $h=4\mu s$}
\end{figure}
\begin{itemize}
	\item The first plot shows the voltages $e_2$ and $e_3$ under inductive load conditions with a timestep (h) of $4x10^{-05}$ seconds. The Voltage $e_2$ levels off at around 13.7V and the Voltage $e_3$ levels off at around 2.3V. The leveling off of voltages $e_2$ and $e_3$ is due to the steady-state reached in the power supply circuit. The voltage $e_2$ levels off at around 13.7V, indicating the equilibrium reached in the capacitor's charging and discharging processes. Likewise,The voltage $e_3$ levels off at around 2.3V, suggesting that the zener diode is biased in the reverse direction and has reached a steady-state voltage.
    \item The second plot displays the currents $I_z$ and $I_L$ under the resistive load conditions. Current $I_L$ represents the inductor current. It initially increases due to changing voltage across the inductor but eventually reaches a steady-state level. The back EMF balances the applied voltage, resulting in a constant current flow. In the plot, $I_z$ levels off at approximately 0.037A, indicating the equilibrium in the inductor's behavior. Current $I_z$ represents the zener diode current. It remains at 0A throughout the simulation because the zener diode is in reverse bias when the voltage across it is below the zener voltage ($V_z$). The zener diode blocks significant current flow, acting as an open circuit. The voltage $e_3$ remains below $V_z$, maintaining the zener diode in a non-conducting state.
\end{itemize}

\subsubsection{Resistive Load}
\begin{figure}[H]
    \centering
    \includegraphics[width=\textwidth]{/graphics/exports/psp_ResistiveLoad_h_4e-05.png}
    \caption{Node voltages and currents under Resistive Load conditions with $h=4\mu s$}
\end{figure}
\begin{itemize}
	\item The first plot shows the voltages $e_2$ and $e_3$ under resistive load conditions with a timestep (h) of $4x10^{-05}$ seconds. The Voltage $e_2$ levels off at around 13.7V and the Voltage $e_3$ levels off at around 6.7V. The leveling off of voltages $e_2$ and $e_3$ is due to the steady-state reached in the power supply circuit. The voltage $e_2$ levels off at around 13.7V, indicating the equilibrium reached in the capacitor's charging and discharging processes. Likewise,The voltage $e_3$ levels off at around 6.7V, suggesting that the zener diode is biased in the reverse direction and has reached a steady-state voltage.
    \item The second plot displays the currents $I_z$ and $I_L$ under the resistive load conditions. Current $I_L$ represents the inductor current. It initially increases due to changing voltage across the inductor but eventually reaches a steady-state level. The back EMF balances the applied voltage, resulting in a constant current flow. In the plot, $I_z$ levels off at approximately 0.037A, indicating the equilibrium in the inductor's behavior. Current $I_z$ represents the zener diode current. It remains at 0A throughout the simulation because the zener diode is in reverse bias when the voltage across it is below the zener voltage ($V_z$). The zener diode blocks significant current flow, acting as an open circuit. The voltage $e_3$ remains below $V_z$, maintaining the zener diode in a non-conducting state.
\end{itemize}

\subsubsection{Full Load}
\begin{figure}[H]
    \centering
    \includegraphics[width=\textwidth]{/graphics/exports/psp_FullLoad_h_4e-05.png}
    \caption{Node voltages and currents under Full Load conditions with $h=4\mu s$}
\end{figure}
\begin{itemize}
	\item The first plot shows the voltages $e_2$ and $e_3$ under full load conditions with a timestep (h) of $4x10^{-05}$ seconds. The Voltage $e_2$ levels off at around 13.7V and the Voltage $e_3$ levels off at around 2V. The leveling off of voltages $e_2$ and $e_3$ is due to the steady-state reached in the power supply circuit. The voltage $e_2$ levels off at around 13.7V, indicating the equilibrium reached in the capacitor's charging and discharging processes. Likewise,The voltage $e_3$ levels off at around 2V, suggesting that the zener diode is biased in the reverse direction and has reached a steady-state voltage.
    \item The second plot displays the currents $I_z$ and $I_L$ under the full load conditions. Current $I_L$ represents the inductor current. It initially increases due to changing voltage across the inductor but eventually reaches a steady-state level. The back EMF balances the applied voltage, resulting in a constant current flow. In the plot, $I_L$ levels off at approximately 0.031A, indicating the equilibrium in the inductor's behavior. Current $I_z$ represents the zener diode current. It remains at 0A throughout the simulation because the zener diode is in reverse bias when the voltage across it is below the zener voltage ($V_z$). The zener diode blocks significant current flow, acting as an open circuit. The voltage $e_3$ remains below $V_z$, maintaining the zener diode in a non-conducting state.

    
\end{itemize}

\pagebreak
\subsection{Maximum Diode Current}
In Section \ref{determinationOfMaximimumCurrent}, the maximum current flowing through the diodes was estimated to be $\simeq 1.917A$.The simulated diode current across all modes of operation is plot for 8 cycles of the input voltage $V_{in}(t)$:
\begin{figure}[H]
	\centering
	\includegraphics[width=\textwidth]{/graphics/exports/diode_current_MIN.png}
	\caption{Maximum diode in the circuit across all modes of operation}
\end{figure}

\begin{itemize}
	\item The maximum current flowing through the diode occurs $t \simeq 0$. This supports the initial assumption made in section \ref{determinationOfMaximimumCurrent}, that the peak current occurs during the first half cycle.
	\item The simulated maximum was $1.92A$ which approximately equals the analytically derived $1.917A$. 
	\item This validates the piece-wise linear approximation of the ideal diode that was made during the modelling of the circuit, and shows that our estimation was very reasonable. 
\end{itemize}

\pagebreak
\subsection{Power Dissipation in $R_s$}
Resistor $R_s$ has a tolerance of $\pm 10\%$ and a maximum power rating of $0.5W$. The power dissipated across resistor $R_s$ must not exceed this rating across all modes of operation. The power is be determined using $P=VI$ and applying Ohm's law such that
\begin{equation}
	P=R_sI_s^2
\end{equation}
The power dissipation was simulated for all four modes of operation with respect to time, for $R_s = [270\Omega, 300\Omega, 330\Omega]$. The maximum power does not exceed the $0.5W$ rating at any point during the simulation, across all four modes with the three possible values of $R_s$.  The results are detailed below

\subsubsection{Minimal Resistance}
\begin{table}[H]
\centering
\begin{tabular}{|c|c|}\hline
	\textbf{Mode of Operation} & \textbf{Maximum Power Dissipated (W)} \\\hline
	No Load & 0.17W \\
	Full Load &  0.17W \\
	Inductive Load & 0.39W \\
	Resistive Load & 0.40W \\\hline
\end{tabular}
\caption{Maximum power dissipated in $R_s$ with $R_s = 270\Omega$ across all four modes}
\end{table}
\begin{figure}[H]
	\centering
	\includegraphics[width=14cm]{/graphics/exports/power_dissipation_min.png}
	\caption{The power dissipated in $R_s$ with $R_s = 270\Omega$ across all four modes}
\end{figure}

\subsubsection{Nominal Resistance}
\begin{table}[H]
\centering
\begin{tabular}{|c|c|}\hline
	\textbf{Mode of Operation} & \textbf{Maximum Power Dissipated (W)} \\\hline
	No Load & 0.19W \\
	Full Load &  0.19W \\
	Inductive Load & 0.43W \\
	Resistive Load & 0.44W \\\hline
\end{tabular}
\caption{Maximum power dissipated in $R_s$ with $R_s = 300\Omega$ across all four modes}
\end{table}
\begin{figure}[H]
	\centering
	\includegraphics[width=14cm]{/graphics/exports/power_dissipation_nominal.png}
	\caption{The power dissipated in $R_s$ with $R_s = 300\Omega$ across all four modes}
\end{figure}

\subsubsection{Maximal Resistance}
\begin{table}[H]
\centering
\begin{tabular}{|c|c|}\hline
	\textbf{Mode of Operation} & \textbf{Maximum Power Dissipated (W)} \\\hline
	No Load & 0.21W \\
	Full Load &  0.21W \\
	Inductive Load & 0.48W \\
	Resistive Load & 0.48W \\\hline
\end{tabular}
\caption{Maximum power dissipated in $R_s$ with $R_s = 330\Omega$ across all four modes}
\end{table}

\begin{figure}[H]
	\centering
	\includegraphics[width=14cm]{/graphics/exports/power_dissipation_max.png}
	\caption{The power dissipated in $R_s$ with $R_s = 330\Omega$ across all four modes}
\end{figure}