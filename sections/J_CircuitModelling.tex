\subsection{Circuit Modelling}
Reconfigure the circuit in Figure \ref{fig:powersupply} with the following adjustments:
\begin{itemize}
	\item The original circuit is taken from the LV side of the ideal transformer, thus, the AC voltage source is given by $V_2$ (referring to the ideal transformer model defined in Equation \ref{eq:ideal_transformer})
	\item The bridge rectifier is taken as two half-rectifiers. One for each half of the cycle (.i.e. half the period of the input sinusoid)
	\item The diodes have been replaced by their linearised models
\end{itemize}

\begin{figure}[H]
	\centering
	
	\begin{circuitikz} [american voltages] \draw
    
    (0,0) to[vsourcesin, l=$V_{2}$] node[label=\textcircled{1}] {} (0,5)
    (0,5) to[battery1, l=$2V_{th}$] (1.9,5)
    (1.5,5) to[R,l=$2R_b$,i=$i_d$, v=$V_d$] (4.5,5)
    (4.5,5) to[nos,o-o] (6,5)
    
    (6,5) node[label=\textcircled{2}] {} to[capacitor, i=$i_c$, l=$C$,v=$V_c$] (6,0)
    
    (6,5) to[R,l=$R_s$,i=$i_s$] (9,5)
    
    (9,5) node[label=\textcircled{3}] {} to[battery1, l=$V_z$] (9,4)
    (9,4) to[R, l=$R_z$, i=$i_z$] (9,2)
    (9,2) to[nos,o-o] (9,0.5)
    (9,0.5) to[short] (9,0)
    
    (9,5) to[short] (12,5)
    (12,5) node[label=\textcircled{4}] {} to[short] (12,4)
    (12,4) to[nos,o-o] (12,3)
    (12,3) to[R, l=$287\Omega$] (12,1)
    (12,1) to[short] (12,0)
    
    (12,5) to[short] (13,5)
    (13,5) to[nos,o-o] (14,5)
    (14,5) to[short] (15,5)
    
    (15,5) node[label=\textcircled{5}] {} to[R,l=$65.7 \Omega$,i=$i_l$] (15,3)
    (15,3) to[L,l=$105.8\text{mH}$] (15,0)
    
    (0,0) to[short] (15,0)
    (7.5,0) node[label=\textcircled{0}] {} node[ground]{} (5,-1);
    
    \end{circuitikz}
	
	\label{circ:multimode}
	\caption{Reconfigured power supply circuit with the multi-mode load}
\end{figure}

\subsubsection{Primary Currents}
The primary currents $i_z$ and $i_d$ are dependant on the unit step function $u(t)$ as earlier discussed.
\paragraph{Diode Current $i_d(t)$} Nodal analysis at \textcircled{2} gives us an expression for $i_d$ in terms of $u(t)$,

\begin{itemize}
	\item The condition for the switch between \textcircled{1} and \textcircled{2} being activated is when
	\begin{equation}
 		\begin{split}
 			V_d > 2V_{th} \\
 			\therefore V_d - 2V_{th} = 0
 		\end{split}
 		\label{eq:ustep_condition}
 	\end{equation}
	\item Thus, the unit step function is unity at $u(V_d > 2V_{th})$
	\item $V_d$ is determined by considering the voltage drop across the two nodes \textcircled{1} and \textcircled{2} $$V_d = e_1 - e_2 - 2V_{th}$$
	\item The current $i_d(t)$ entering the switch is given by $\frac{V_d}{2R_b}$:
	\begin{equation}
		\begin{split}
			i_d(t) &= \frac{e_1 - e_2 - 2V_{th}}{2R_b} \\
				   &= \frac{\frac{e_1 - e_2}{2} - V_{th}}{R_b}	
		\end{split}
	\end{equation}
\end{itemize} 
Thus, to account for the switch we must multiply $i_d(t)$ by the unit step function $u(t)$ that is active when the condition in Equation \ref{eq:ustep_condition} is met:
\begin{equation}
	i_d(t) = \sbracket{\frac{\frac{e_1 - e_2}{2} - V_{th}}{R_b}}\cdot u\bracket{\frac{e_1 - e_2}{2} - V_{th}}
	\label{eq:diode_current}
\end{equation}

\paragraph{Zener Diode Current $i_z(t)$}
Likewise, the current flowing through the Zener can be determined by considering nodes \textcircled{2} and \textcircled{3}

\begin{itemize}
	\item The condition for the zener diode switch is activated when \begin{equation}
		\begin{split}
			e_3 > V_z \\
			\therefore e_3 - V_z = 0
			\label{eq:zener_unity_condition}
		\end{split}
	\end{equation}
	\item The current $i_z(t)$ entering the switch is given by $\frac{V_z}{R_z}$
	\begin{equation}
		i_z(t) = \frac{e_3 - V_z}{R_z}
	\end{equation}
\end{itemize}
Thus, to account for the switch, multiply $i_z(t)$ by $u(t)$ so that the condition in Eq. \ref{eq:zener_unity_condition} is met.
\begin{equation}
    i_z = \sbracket{\frac{e_3 - V_z}{R_z}}\cdot u\bracket{e_3 - V_z}
    \label{eq:zener_current}
\end{equation}

\paragraph{Node Definitions and Voltage Terms}
Within the subsequent analysis, the node voltages $e_1$ and $e_2$ are frequently utilised.
\begin{itemize}
	\item Here, $e_1$ represents the AC input voltage, $V_2$, while $e_2$ indicates the voltage at node \textcircled{2}, where the capacitor is situated.
	\item The expression $\frac{e_1 - e_2}{2}$ denotes half the voltage difference between the AC input voltage and the node voltage at \textcircled{2}.
	\item The relevance of this difference lies in its critical role in determining the operational mode of the circuit as well as the states of various circuit elements such as the diode and the Zener diode.
\end{itemize}

\pagebreak
\subsection{Modes of Operation}
The circuit operates on several modes determined by the load equivalent circuit model in Figure \ref{fig:multimode_load},
\begin{itemize}
	\item Mode 1: Empty load (.i.e. both switches are turned off)
	\item Mode 2: Inductive load (.i.e. the primary switch is turned off)
	\item Mode 3: Resistive load (.i.e. the secondary switch is turned off) 
	\item Mode 4: Full load (.i.e. all switches are turned on)
\end{itemize}
A set of differential equations describing the operation of mode is determined through nodal analysis

\import{sections/circuit_models/}{NoLoad.tex}
\pagebreak \import{sections/circuit_models/}{InductiveLoad.tex}
\pagebreak \import{sections/circuit_models/}{ResistiveLoad.tex}
\pagebreak \import{sections/circuit_models/}{FullLoad.tex}

\iffalse
\subparagraph{Reverse bias}
\subparagraph{Forward Bias}
When $\frac{R_{L_1}e_2}{R_s+R_{L_1}} \geq V_z$, the Zener diode operates in the Zener region, and the system is governed by the following set of equations:

\begin{equation}
    \begin{cases}
        u(\frac{V_2-e_2}{2}-V_{th})(\frac{\frac{V_2-e_2}{2}-V_{th}}{R_b}=i_c+i_s\\
        i_c=C\frac{de_2}{dt}\\
        i_s=\frac{e_2-e_3}{R_s}=\frac{e_3}{R_{L_1}}
    \end{cases}
    \label{eq:system_zener}
\end{equation}

Substituting the value of $e_3$ from Equation \ref{eq:system_zener} yields:

\begin{align}
    \frac{de_2}{dt} &= \frac{1}{C} \times \left[ u\left(\frac{V_2-e_2}{2}-V_{th}\right) \times \left(\frac{\frac{V_2-e_2}{2}-V_{th}}{R_b}\right) - \frac{e_2}{R_s+R_{L_1}} \right] \\
    &= \frac{1}{C} \times \left[ u\left(\frac{V_2-e_2}{2}-V_{th}\right) \times I_{d2} - I_{s2} \right], \text{if } \frac{R_{L_1}e_2}{R_s+R_{L_1}} \geq V_z
\end{align}
%Therefore, it is evident that when a resistor is connected, the current flowing through resistor Rs can be expressed as:
%\begin{equation}
%    I_{R_s} = \frac{e_2 - e_3}{R_s}
%\end{equation}
%
%
%where $V_{in}$ is the input voltage and $e_2$ is the voltage across the load resistor.\\
%
%\begin{equation}
%    i_s=\begin{cases}
%        \frac{e_2-e_3}{R_s}=\frac{e_3}{R_L},  e_3<V_z\\
%        \frac{e_2-e_3}{R_s}=\frac{e_3-V_z}{R_z}+\frac{e_3}{R_L},  e_3 \geq V_z
%    \end{cases}
%\end{equation}\\
%
%When $e_3$ < $V_z$, the voltage $e_3$ can be expressed as $\frac{R_Le_2}{R_s+R_L}$. Therefore, the above expression can be equivalently written as:
%
%\begin{equation}
%    \begin{cases}
%        \frac{e_2-e_3}{R_s}=\frac{e_3}{R_L}, \frac{R_Le_2}{R_s+R_L}<V_z\\
%        \frac{e_2-e_3}{R_s}=\frac{e_3-V_z}{R_z}+\frac{e_3}{R_L}, \frac{R_Le_2}{R_s+R_L} \geq V_z
%    \end{cases}
%\end{equation}\\
%
%Hence, if $\frac{R_Le_2}{R_s+R_L}<V_z$, the Zener diode is in cut-off mode. In this case, the system can be described by the following expressions:
%
%\begin{equation}
%    \begin{cases}
%        u(\frac{V_2-e_2}{2}-V_{th})(\frac{\frac{V_2-e_2}{2}-V_{th}}{R_b}=i_c+i_s\\
%        i_c=C\frac{de_2}{dt}\\
%        i_s=\frac{e_2-e_3}{R_s}=\frac{e_3-V_z}{R_z}+\frac{e_3}{R_L}
%    \end{cases}
%\end{equation}\\
%
%Therefore, the voltage across the load, Vb, can be determined as:
%
%\begin{equation}
%    e_3=\frac{R_LR_ze_2+R_LR_sV_z}{R_zR_s+R_zR_L+R_sR_L}
%\end{equation}
%
%The following expression determines the current flowing through the resistor Rs:
%
%\begin{equation}
%    i_s=\frac{e_2-e_3}{R_s}=\frac{e_2(\frac{1}{R_z}+\frac{1}{R_L})-\frac{V_z}{R_z}}{(\frac{1}{R_s}+\frac{1}{R_z}+\frac{1}{R_L})R_s}=\frac{e_2(\frac{R_L+R_z}{R_zR_L})-\frac{R_LV_z}{R_LR_z}}{(\frac{R_zR_L+R_sR_z+R_sR_L}{R_zR_L})}=\frac{e_2(R_L+R_z)-R_LV_z}{(R_zR_L+R_sR_z+R_sR_L)}
%\end{equation}\\
%
%As a result, the current through the capacitor can be calculated as:
%
%\begin{equation}
%    i_c=i_d-i_s=u(\frac{V_2-e_2}{2}-V_{th})(\frac{\frac{V_2-e_2}{2}-V_{th}}{R_b}-\frac{e_2(R_L+R_z)-R_LV_Z}{(R_zR_L+R_s(R_z+R_L))}
%\end{equation}\\
%
%Consequently, the resulting differential equation can be expressed as:\\
%
%\begin{equation}
%    \frac{de_2}{dt}=\frac{1}{C}\left(u\left(\frac{V_2-e_2}{2}-V_{th}\right)\left(\frac{\frac{V_2-e_2}{2}-V_{th}}{R_b}\right)-\frac{e_2(R_L+R_z)-R_LV_Z}{R_zR_L+R_s(R_z+R_L)}\right), \text{if } \frac{R_Le_2}{R_s+R_L} \geq V_z
%\end{equation}\\

%% MODE 3

\large\textcolor{red}{Mode 3: Taking into account the presence of both a resistor and an inductor at the load:}\\

[insert figure 2.1]\\

\emph{Figure 2.1 illustrates the simplified circuit configuration when a resistor and an inductor are connected at the output, with one device activated.}

\textcolor{blue}{KCL at Node A:}\\
\begin{equation}
    i_d=i_c+i_s
\end{equation}
\begin{equation}
    u(\frac{V_2-e_2}{2}-V_{th})(\frac{\frac{V_2-e_2}{2}-V_{th}}{R_b}=C(\frac{de_2}{dt}+\frac{e_2-V_b}{R_s}
\end{equation}\\

\textcolor{blue}{KCL at Node B:}\\
\begin{equation}
    i_s=i_z+i_L
\end{equation}
\begin{equation}
    \frac{e_2-V_b}{R_s}=\frac{V_b-V_z}{R_z}u(V_b-V_z)+i_L
\end{equation}\\

\textcolor{blue}{KCL at Node C:}\\
\begin{equation}
    i_L=\frac{V_b-V_c}{R_2}
\end{equation}\\

Therefore, it is evident that when a resistor and an inductor are connected, the current flowing through the resistor Rs can be determined by the following expression:

\begin{equation}
     i_s=\begin{cases}
        \frac{e_2-V_b}{R_s}=i_L,  V_b<V_z\\
        \frac{e_2-V_b}{R_s}=\frac{V_b-V_z}{R_z}+i_L,  V_b \geq V_z
        \end{cases}
\end{equation}\\

When $V_b < V_z$, $e_2 - R_sI_L$ is equal to $V_b$. Therefore, the above expression can be equivalently written as follows:

\begin{equation}
    \begin{cases}
        \frac{e_2-V_b}{R_s}=\frac{V_b}{R_L}, e_2-R_sI_L<V_z\\
        \frac{e_2-V_b}{R_s}=\frac{V_b-V_z}{R_z}+i_L, e_2-R_sI_L>V_z
    \end{cases}
\end{equation}\\

Thus, the Zener is in cut-off if $e_2 - R_sI_L < V_z$, and the resulting expressions describing the system are:

\begin{equation}
    \begin{cases}
        u(\frac{V_2-e_2}{2}-V_{th})(\frac{\frac{V_2-e_2}{2}-V_{th}}{R_b})=i_s+i_L\\
        i_c=C\frac{de_2}{dt}\\
        V_c=L\frac{di_L}{dt}\\
        i_L=i_s=\frac{V_b-V_c}{R_2}=\frac{e_2-V_b}{R_s}
    \end{cases}
\end{equation}\\

Therefore, the voltage across the inductor can be determined as:\\

\begin{equation}
    V_c=e_2-(R_s+R_2)i_L
\end{equation}

Furthermore, the current flowing through the capacitor can be determined as:\\

\begin{equation}
    i_c=i_d-i_L=u(\frac{V_2-e_2}{2}-V_{th})(\frac{V_2-e_2}{2}-V_{th})(\frac{\frac{V_2-e_2}{2}-V_{th}}{R_b})-i-L
\end{equation}\\

Consequently, the resulting system of differential equations can be expressed as:\\

\begin{equation}
    \begin{cases}
        \frac{di_L}{dt}=\frac{1}{L}(e_2-(R_s+R_2)i_L)\\
        \frac{de_2}{dt}=\frac{1}{C}\left(u\left(\frac{V_2-e_2}{2}-V_{th}\right)\left(\frac{V_2-e_2}{2}-V_{th}\right)\left(\frac{\frac{V_2-e_2}{2}-V_{th}}{R_b}\right)-i_L\right), \text{if } e_2-R_si_L<V_z
    \end{cases}
\end{equation}\\

Thus, the Zener is in reverse bias if $e_2 - R_sI_L \geq V_z$, and the resulting expressions describing the system are:

\begin{equation}
    \begin{cases}
        u(\frac{V_2-e_2}{2}-V_{th})(\frac{\frac{V_2-e_2}{2}-V_{th}}{R_b})=i_s+i_L\\
        i_c=C\frac{de_2}{dt}\\
        V_c=L\frac{di_L}{dt}\\
        i_s=\frac{e_2-V_b}{R_s}=\frac{V_b-V_z}{R_z}+i_L\\
        V_c=V_b-R_2i_L
    \end{cases}
\end{equation}\\

Therefore, the output voltage, Vb, can be expressed as:\\

\begin{equation}
    V_b=\frac{R_ze_2+R_sV_z-i_LR_sR_z}{R_z+R_s}
\end{equation}\\

Additionally, the voltage across the inductor, denoted as Vc, can also be determined:\\

\begin{equation}
    V_c=V_b-R_2i_L=\frac{R_ze_2+R_s(V_z-i_LR_z)}{R_z+R_s}-R_2i_L=\frac{R_ze_2+R_sV_z-i_L(R_s(R_z+R_2)+R_2R_z)}{R_z+R_s}
\end{equation}\\

The current flowing through the capacitor, ic, can be calculated as:\\

\begin{equation}
    i_c=i_d-i_L=u(\frac{V_2-e_2}{2}-V_{th})(\frac{V_2-e_2}{2}-V_{th})(\frac{\frac{V_2-e_2}{2}-V_{th}}{R_b})-\frac{e_2-V_z+i_LR_z}{R_z+R_s}
\end{equation}\\

Thus, the resulting set of differential equations is:

\begin{equation}
    \begin{cases}
        \frac{di_L}{dt}=\frac{1}{L}(\frac{R_ze_2+R_sV_z-i_L(R_s(R_z+R_2)+R_2R_z)}{(R_z+R_s)})\\
        \frac{de_2}{dt}=\frac{1}{C}(u(\frac{V_2-e_2}{2}-V_{th})(\frac{V_2-e_2}{2}-V_{th})(\frac{\frac{V_2-e_2}{2}-V_{th}}{R_b})-\frac{e_2-V_z+i_LR_z}{R_z+R_s}),  \text{if } e_2-R_si_L \geq V_z
    \end{cases}
\end{equation}\\

\large\textcolor{red}{Mode 4: Taking into account the scenario where both devices are activated, the circuit depicted in Figure (1.9) can be simplified as follows:}\\

[insert figure 2.2]\\

\emph{Figure 2.2 illustrates the simplified circuit configuration when a resistor load, resistor R, and an inductor are connected at the output. In this case, both devices are activated.}\\

\textcolor{blue}{KCL at Node A:}\\
\begin{equation}
    i_d=i_c+i_s
\end{equation}
\begin{equation}
    u(\frac{V_2-e_2}{2}-V_{th})(\frac{\frac{V_2-e_2}{2}-V_{th}}{R_b})=C(\frac{de_2}{dt}+\frac{e_2-V_b}{R_s})
\end{equation}\\

\textcolor{blue}{KCL at Node B:}\\
\begin{equation}
    i_s=i_z+i_L+i_{R_L}
\end{equation}
\begin{equation}
    \frac{e_2-V_b}{R_s}=\frac{V_b-V_z}{R_z}u(V_b-V_z)+i_L+\frac{V_b}{R_L}
\end{equation}\\

\textcolor{blue}{KCL at Node C:}\\
\begin{equation}
    i_L=\frac{V_b-V_c}{R_2}
\end{equation}\\

As a result of connecting a resistor and an inductor, the current passing through the resistor Rs can be expressed as follows:\\

\begin{equation}
    \begin{cases}
        \frac{e_2-V_b}{R_s}=i_L+\frac{V_b}{R_L},  V_b<V_z\\
        \frac{e_2-V_b}{R_s}=\frac{V_b-V_z}{R_z}+i_L+\frac{V_b}{R_L},  V_b>V_z
    \end{cases}
\end{equation}\\

Therefore, if $V_b < V_z$, the equation $V_b = e_2 - \frac{i_L R_s}{R_s+R_L} R_L$ can be alternatively expressed as:\\

\begin{equation}
    \begin{cases}
        \frac{e_2-V_b}{R_s}=i_L+\frac{V_b}{R_L},  \frac{e_2-i_LR_s}{R_s+R_L}R_L<V_z\\
        \frac{e_2-V_b}{R_s}=\frac{V_b-V_z}{R_z}+i_L+\frac{V_b}{R_L},  \frac{e_2-i_LR_s}{R_s+R_L}>V_z
    \end{cases}
\end{equation}\\

Therefore, the Zener diode is in the cut-off state if $e_2 - \frac{i_L R_s}{R_s+R_L} R_L < V_Z$. In this case, the resulting expressions describing the system are:\\


[this below system of equations needs to be check, specifically the = $i_c+i_s$ that eugene has. Should it be $i_s+i_L$ and the value for $i_s $]
\begin{equation}
    \begin{cases}
        u(\frac{V_2-e_2}{2}-V_{th})(\frac{\frac{V_2-e_2}{2}-V_{th}}{R_b})=i_s+i_L\\
        i_s=\frac{e_2-V_b}{R_s}=i_L+\frac{V_b}{R_L}
        i_c=C\frac{de_2}{dt}\\
        V_c=L\frac{di_L}{dt}\\
        i_s=\frac{e_2-V_b}{R_s}=\frac{V_b-V_z}{R_z}+i_L\\
        V_c=V_b-R_2i_L
    \end{cases}
\end{equation}\\

Therefore, the output voltage, Vb, can be expressed as:\\

\begin{equation}
    V_b=\frac{V-a-i_LR_s}{R_s+R_L}R_L
\end{equation}\\

Therefore, the voltage across the inductor can be described as:\\

\begin{equation}
    V_c=V_b-R_2i_L=\frac{e_2-i_LR_s}{R_s+R_L}R_L-R_2i_L
\end{equation}\\

Moreover, the current flowing through the resistor Rs can be expressed as:\\

\begin{equation}
    i_s=i_L+\frac{V_b}{R_L}=i_L+\frac{e_2-i_LR_s}{R_s+R_L}=\frac{e_2+i_LR_L}{R_s+R_L}
\end{equation}\\

Furthermore, the current flowing through the capacitor can be described as:\\

\begin{equation}
    i_c=i_d-i_s=u(\frac{V_2-e_2}{2}-V_{th})(\frac{\frac{V_2-e_2}{2}-V_{th}}{R_b})-\frac{e_2+i_LR_L}{R_s+R_L}
\end{equation}\\

Consequently, the resulting set of differential equations can be expressed as:\\

\begin{equation}
    \begin{cases}
        \frac{di_L}{dt}=\frac{1}{L}(\frac{e_2R_L-i_L(R_s(R_L+R_2)+R_LR_2)}{(R_L+R_s)})\\
        \frac{de_2}{dt}=\frac{1}{C}(u(\frac{V_2-e_2}{2}-V_{th})(\frac{\frac{V_2-e_2}{2}-V_{th}}{R_b})-\frac{e_2+i_LR_L}{R_L+R_s}),  \text{if } \frac{e_2-i_LR_s}{R_s+R_L}R_L<V_z
    \end{cases}
\end{equation}\\

In the event that the Zener diode is subjected to reverse bias, and the condition $e_2 - \frac{i_L R_s}{R_s+R_L} R_L \geq V_Z$ is satisfied, the system can be described by the following expressions:\\

\begin{equation}
    \begin{cases}
        u(\frac{V_2-e_2}{2}-V_{th})(\frac{\frac{V_2-e_2}{2}-V_{th}}{R_b})=i_s+i_L\\
        i_s=\frac{e_2-V_b}{R_s}=i_L+\frac{V_b}{R_L}
        i_c=C\frac{de_2}{dt}\\
        V_c=L\frac{di_L}{dt}\\
        V_c=V_b-R_2i_L
    \end{cases}
\end{equation}\\

Therefore, the resulting output voltage, Vb, can be expressed as:\\

\begin{equation}
    V_b=\frac{e_2R_z+V_zR_s-i_LR_sR_z}{R_zR_s+R_LR_z+R_sR_L}R_L
\end{equation}\\

As a result, the voltage across the inductor can be denoted as:\\

\begin{equation}
    V_c=V_b-R_2i_L=\frac{e_2R_z+V_zR_s-i_LR_sR_z}{R_zR_s+R_LR_z+R_sR_L}R_L-R_2i_L
\end{equation}\\

Additionally, the current flowing through resistor Rs can be described as:\\

\begin{equation}
    i_s=\frac{e_2(R_z+R_L)-V_zR_L+i_LR_LR_z}{R_zR_s+R_LR_z+R_sR_L}
\end{equation}\\

Moreover, the current flowing through the capacitor can be expressed as:\\

\begin{equation}
    i_c=i_d-i_s=u(\frac{V_2-e_2}{2}-V_{th})(\frac{\frac{V_2-e_2}{2}-V_{th}}{R_b})-\frac{e_2(R_z+R_L)-V_zR_L+i_LR_zR_L}{R_zR_s+R_LR_z+R_sR_L}
\end{equation}\\

Consequently, the resulting set of differential equations can be summarized as:\\

\begin{equation}
    \begin{cases}
        \frac{di_L}{dt}=\frac{1}{L}(\frac{e_2R_z+V_zR_s-i_LR_sR_z}{R_zR_s+R_LR_z+R_sR_L})\\
        \frac{de_2}{dt}=\frac{1}{C}(u(\frac{V_2-e_2}{2}-V_{th})(\frac{\frac{V_2-e_2}{2}-V_{th}}{R_b})-\frac{e_2(R_z+R_L)-V_zR_L+i_LR_zR_L}{R_zR_s+R_LR_z+R_sR_L}),  \text{if } \frac{e_2-i_LR_s}{R_s+R_L}R_L \geq V_z
    \end{cases}
\end{equation}\\
\fi


