\subsection{Circuit Modelling}
Based on the previously discussed assumptions and Figure (1.9), the circuit can be reconfigured as depicted in Figure (2). To elucidate the performance of the circuit in all four load modes, the following equations can be employed:\\

[insert circuit diagram 1.7]\\

\emph{Figure 1.7: Power supply circuit with multimode load $R_L$}

[insert circuit diagram 1.8]\\

\emph{Figure 1.8: Modified circuit with implemented model of circuit elements}\\

The current analysis of the diodes involves utilizing the unit step function, which acts as a switch in the model. When in the on-state, the unit step function is set to 1, and when in the off-state, it is set to 0. This function can be defined as follows:

\begin{equation}
    i_d=u(\frac{V_2-V_a}{2}-V_t)(\frac{\frac{V_2-Va}{2}-V_t}{R_d})
\end{equation}\\

Likewise, the current flowing through the Zener diode:\\

\begin{equation}
    i_z=(\frac{V_b-V_z}{R_z})[u(V_b-V_z)]
\end{equation}\\

\large\textcolor{red}{Mode 1: The load does not include an inductor or resistor (both components are deactivated)}:\\

[insert figure 1.9]\\

\emph{Figure 1.9: Circuit with no load (both devices off)}\\

\textcolor{blue}{KCL at node A:}

\begin{equation}
    i_d=i_c+i_s
    u(\frac{V_2-V_a}{2}-V_t)(\frac{\frac{V_2-Va}{2}-V_t}{R_d})=C(\frac{dV_a}{dt}+\frac{V_a-V_b}{R_s}
\end{equation}\\

where u represents the unit step response.

\textcolor{blue}{KCL at Node B:}
\begin{equation}
    i_s=i_z
    (\frac{V_a-V_b}{R_s}=\frac{V_b-V_z}{R_z}u(V_b-V_z)
\end{equation}\\

Therefore, it is evident that when no load is connected, the current through the Zener can be determined by:

\begin{equation}
    i_z= \begin{cases}
    \frac{V_a-V_b}{R_s}=0,  V_b<V_z\\
    \frac{V_a-V_b}{R_s}=\frac{V_b-V_z}{R_z},  V_b\geq V_z
    \end{cases}
\end{equation}\\

When $V_b$ < $V_z$, $V_a$ is equal to $V_b$, and in this scenario, the current through the Zener can be expressed as:

\begin{equation}
    i_z=\begin{cases}
        0,  V_a<V_z \\
        \frac{V_a-V_b}{R_s}=\frac{V_b-V_z}{R_z},  V_a>V_z
    \end{cases}
\end{equation}

Therefore, if $V_a < V_z$, the Zener diode is in the cut-off region. In this case, the system can be described by the following expressions:\\

\begin{equation}
    \begin{cases}
        i_c=C\frac{dV_a}{dt}\\
        u(\frac{V_2-V_a}{2}-V_{th})(\frac{\frac{V_2-V_a}{2}-V_{th}}{R_d})=i_c
    \end{cases}
\end{equation}\\

Consequently, the differential equation that arises from this is:

\begin{equation}
    \frac{dV_a}{dt}=\frac{1}{C}u(\frac{V_2-V_a}{2}-V_{th})(\frac{\frac{V_2-V_a}{2}-V_{th}}{R_d}),  \emph{if $V_a<V_z$}
\end{equation}

When $V_a \geq V_z$, the Zener diode is in reverse bias. In this situation, the system can be characterized by the following expressions:

\begin{equation}
    \begin{cases}
        i_c=C\frac{dV_a}{dt}\\
        u(\frac{V_2-V_a}{2}-V_{th})(\frac{\frac{V_2-V_a}{2}-V_{th}}{R_d})=i_c+i_s\\
        i_s=\frac{V_a-V_b}{R_s}=\frac{V_b-V_z}{R_z}
    \end{cases}
\end{equation}\\

Thus, it is possible to determine the output voltage, denoted as $V_b$:\\

\begin{equation}
    V_b=\frac{V_aR_z+V_zR_s}{R_s+R_z}
\end{equation}

As a consequence, the following expression describes the current flowing through resistor $R_s$, which can be calculated as:

\begin{equation}
    i_s=\frac{V_a-V_b}{R_s}=\frac{V_a-\frac{V_aR_z+V_zR_s}{R_s+R_z}}{R_s}=\frac{V_a(R_s+R_z)-V_aR_z-V_zR_s}{R_s(R_s+R_z)}=\frac{V_a-V_z}{(R_s+R_z)}
\end{equation}

The current through the capacitor can now be calculated using the following expression:\\

\begin{equation}
    i_c=u(\frac{V_2-V_a}{2}-V_t)(\frac{\frac{V_2-V_a}{2}-V_{th}}{R_d})-\frac{V_a-V_z}{(R_s+R_z)}
\end{equation}


As a result, the differential equation that emerges from this is:

\begin{equation}
    \frac{dV_a}{dt}=\frac{1}{C}\left(u\left(\frac{V_2-V_a}{2}-V_t\right)\left(\frac{\frac{V_2-V_a}{2}-V_{th}}{R_d}\right)-\frac{V_a-V_z}{R_s+R_z}\right), \text{if } V_a \geq V_z
\end{equation}\\

\large\textcolor{red}{\emph{Mode 2: When a resistor is present at the load, the circuit depicted in Figure (1.9) can be simplified as follows:}}\\

[insert figure 2]\\

\emph{Figure 2 illustrates the revised circuit configuration when a resistor load is connected at the output, with one device activated.}\\

\textcolor{blue}{KCL at Node A:}\\
\begin{equation}
    i_d=i_c+i_s
\end{equation}
\begin{equation}
    u(\frac{V_2-V_a}{2}-V_{th})(\frac{\frac{V_2-V_a}{2}-V_{th}}{R_d}=C(\frac{dV_a}{dt}+\frac{V_a-V_b}{R_s}
\end{equation}\\

\textcolor{blue}{KCL at Node B:}\\
\begin{equation}
    i_s=i_z+i_L
\end{equation}
\begin{equation}
    \frac{V_a-V_b}{R_s}=\frac{V_b-V_z}{R_z}u(V_b-V_z)+\frac{V_b}{R_L}
\end{equation}\\

Therefore, it is evident that when a resistor is connected, the current flowing through resistor Rs can be expressed as:
\begin{equation}
    I_{R_s} = \frac{V_{in} - V_a}{R_s}
\end{equation}


where $V_{in}$ is the input voltage and $V_a$ is the voltage across the load resistor.\\

\begin{equation}
    i_s=\begin{cases}
        \frac{V_a-V_b}{R_s}=\frac{V_b}{R_L},  V_b<V_z\\
        \frac{V_a-V_b}{R_s}=\frac{V_b-V_z}{R_z}+\frac{V_b}{R_L},  V_b \geq V_z
    \end{cases}
\end{equation}\\

When $V_b$ < $V_z$, the voltage $V_b$ can be expressed as $\frac{R_LV_a}{R_s+R_L}$. Therefore, the above expression can be equivalently written as:

\begin{equation}
    \begin{cases}
        \frac{V_a-V_b}{R_s}=\frac{V_b}{R_L}, \frac{R_LV_a}{R_s+R_L}<V_z\\
        \frac{V_a-V_b}{R_s}=\frac{V_b-V_z}{R_z}+\frac{V_b}{R_L}, \frac{R_LV_a}{R_s+R_L} \geq V_z
    \end{cases}
\end{equation}\\

Hence, if $\frac{R_LV_a}{R_s+R_L}<V_z$, the Zener diode is in cut-off mode. In this case, the system can be described by the following expressions:

\begin{equation}
    \begin{cases}
        u(\frac{V_2-V_a}{2}-V_{th})(\frac{\frac{V_2-V_a}{2}-V_{th}}{R_d}=i_c+i_s\\
        i_c=C\frac{dV_a}{dt}\\
        i_s=\frac{V_a-V_b}{R_s}=\frac{V_b-V_z}{R_z}+\frac{V_b}{R_L}
    \end{cases}
\end{equation}\\

Therefore, the voltage across the load, Vb, can be determined as:

\begin{equation}
    V_b=\frac{R_LR_zV_a+R_LR_sV_z}{R_zR_s+R_zR_L+R_sR_L}
\end{equation}

The following expression determines the current flowing through the resistor Rs:

\begin{equation}
    i_s=\frac{V_a-V_b}{R_s}=\frac{V_a(\frac{1}{R_z}+\frac{1}{R_L})-\frac{V_z}{R_z}}{(\frac{1}{R_s}+\frac{1}{R_z}+\frac{1}{R_L})R_s}=\frac{V_a(\frac{R_L+R_z}{R_zR_L})-\frac{R_LV_z}{R_LR_z}}{(\frac{R_zR_L+R_sR_z+R_sR_L}{R_zR_L})}=\frac{V_a(R_L+R_z)-R_LV_z}{(R_zR_L+R_sR_z+R_sR_L)}
\end{equation}\\

As a result, the current through the capacitor can be calculated as:

\begin{equation}
    i_c=i_d-i_s=u(\frac{V_2-V_a}{2}-V_{th})(\frac{\frac{V_2-V_a}{2}-V_{th}}{R_d}-\frac{V_a(R_L+R_z)-R_LV_Z}{(R_zR_L+R_s(R_z+R_L))}
\end{equation}\\

Consequently, the resulting differential equation can be expressed as:\\

\begin{equation}
    \frac{dV_a}{dt}=\frac{1}{C}\left(u\left(\frac{V_2-V_a}{2}-V_{th}\right)\left(\frac{\frac{V_2-V_a}{2}-V_{th}}{R_d}\right)-\frac{V_a(R_L+R_z)-R_LV_Z}{R_zR_L+R_s(R_z+R_L)}\right), \text{if } \frac{R_LV_a}{R_s+R_L} \geq V_z
\end{equation}\\

\large\textcolor{red}{Mode 3: Taking into account the presence of both a resistor and an inductor at the load:}\\

[insert figure 2.1]\\

\emph{Figure 2.1 illustrates the simplified circuit configuration when a resistor and an inductor are connected at the output, with one device activated.}

\textcolor{blue}{KCL at Node A:}\\
\begin{equation}
    i_d=i_c+i_s
\end{equation}
\begin{equation}
    u(\frac{V_2-V_a}{2}-V_{th})(\frac{\frac{V_2-V_a}{2}-V_{th}}{R_d}=C(\frac{dV_a}{dt}+\frac{V_a-V_b}{R_s}
\end{equation}\\

\textcolor{blue}{KCL at Node B:}\\
\begin{equation}
    i_s=i_z+i_L
\end{equation}
\begin{equation}
    \frac{V_a-V_b}{R_s}=\frac{V_b-V_z}{R_z}u(V_b-V_z)+i_L
\end{equation}\\

\textcolor{blue}{KCL at Node C:}\\
\begin{equation}
    i_L=\frac{V_b-V_c}{R_2}
\end{equation}\\

Therefore, it is evident that when a resistor and an inductor are connected, the current flowing through the resistor Rs can be determined by the following expression:

\begin{equation}
     i_s=\begin{cases}
        \frac{V_a-V_b}{R_s}=i_L,  V_b<V_z\\
        \frac{V_a-V_b}{R_s}=\frac{V_b-V_z}{R_z}+i_L,  V_b \geq V_z
        \end{cases}
\end{equation}\\

When $V_b < V_z$, $V_a - R_sI_L$ is equal to $V_b$. Therefore, the above expression can be equivalently written as follows:

\begin{equation}
    \begin{cases}
        \frac{V_a-V_b}{R_s}=\frac{V_b}{R_L}, V_a-R_sI_L<V_z\\
        \frac{V_a-V_b}{R_s}=\frac{V_b-V_z}{R_z}+i_L, V_a-R_sI_L>V_z
    \end{cases}
\end{equation}\\

Thus, the Zener is in cut-off if $V_a - R_sI_L < V_z$, and the resulting expressions describing the system are:

\begin{equation}
    \begin{cases}
        u(\frac{V_2-V_a}{2}-V_{th})(\frac{\frac{V_2-V_a}{2}-V_{th}}{R_d})=i_s+i_L\\
        i_c=C\frac{dV_a}{dt}\\
        V_c=L\frac{di_L}{dt}\\
        i_L=i_s=\frac{V_b-V_c}{R_2}=\frac{V_a-V_b}{R_s}
    \end{cases}
\end{equation}\\

Therefore, the voltage across the inductor can be determined as:\\

\begin{equation}
    V_c=V_a-(R_s+R_2)i_L
\end{equation}

Furthermore, the current flowing through the capacitor can be determined as:\\

\begin{equation}
    i_c=i_d-i_L=u(\frac{V_2-V_a}{2}-V_{th})(\frac{V_2-V_a}{2}-V_{th})(\frac{\frac{V_2-V_a}{2}-V_{th}}{R_d})-i-L
\end{equation}\\

Consequently, the resulting system of differential equations can be expressed as:\\

\begin{equation}
    \begin{cases}
        \frac{di_L}{dt}=\frac{1}{L}(V_a-(R_s+R_2)i_L)\\
        \frac{dV_a}{dt}=\frac{1}{C}\left(u\left(\frac{V_2-V_a}{2}-V_{th}\right)\left(\frac{V_2-V_a}{2}-V_{th}\right)\left(\frac{\frac{V_2-V_a}{2}-V_{th}}{R_d}\right)-i_L\right), \text{if } V_a-R_si_L<V_z
    \end{cases}
\end{equation}\\

Thus, the Zener is in reverse bias if $V_a - R_sI_L \geq V_z$, and the resulting expressions describing the system are:

\begin{equation}
    \begin{cases}
        u(\frac{V_2-V_a}{2}-V_{th})(\frac{\frac{V_2-V_a}{2}-V_{th}}{R_d})=i_s+i_L\\
        i_c=C\frac{dV_a}{dt}\\
        V_c=L\frac{di_L}{dt}\\
        i_s=\frac{V_a-V_b}{R_s}=\frac{V_b-V_z}{R_z}+i_L\\
        V_c=V_b-R_2i_L
    \end{cases}
\end{equation}\\

Therefore, the output voltage, Vb, can be expressed as:\\

\begin{equation}
    V_b=\frac{R_zV_a+R_sV_z-i_LR_sR_z}{R_z+R_s}
\end{equation}\\

Additionally, the voltage across the inductor, denoted as Vc, can also be determined:\\

\begin{equation}
    V_c=V_b-R_2i_L=\frac{R_zV_a+R_s(V_z-i_LR_z)}{R_z+R_s}-R_2i_L=\frac{R_zV_a+R_sV_z-i_L(R_s(R_z+R_2)+R_2R_z)}{R_z+R_s}
\end{equation}\\

The current flowing through the capacitor, ic, can be calculated as:\\

\begin{equation}
    i_c=i_d-i_L=u(\frac{V_2-V_a}{2}-V_{th})(\frac{V_2-V_a}{2}-V_{th})(\frac{\frac{V_2-V_a}{2}-V_{th}}{R_d})-\frac{V_a-V_z+i_LR_z}{R_z+R_s}
\end{equation}\\

Thus, the resulting set of differential equations is:

\begin{equation}
    \begin{cases}
        \frac{di_L}{dt}=\frac{1}{L}(\frac{R_zV_a+R_sV_z-i_L(R_s(R_z+R_2)+R_2R_z)}{(R_z+R_s)})\\
        \frac{dV_a}{dt}=\frac{1}{C}(u(\frac{V_2-V_a}{2}-V_{th})(\frac{V_2-V_a}{2}-V_{th})(\frac{\frac{V_2-V_a}{2}-V_{th}}{R_d})-\frac{V_a-V_z+i_LR_z}{R_z+R_s}),  \text{if } V_a-R_si_L \geq V_z
    \end{cases}
\end{equation}\\

\large\textcolor{red}{Mode 4: Taking into account the scenario where both devices are activated, the circuit depicted in Figure (1.9) can be simplified as follows:}\\

[insert figure 2.2]\\

\emph{Figure 2.2 illustrates the simplified circuit configuration when a resistor load, resistor R, and an inductor are connected at the output. In this case, both devices are activated.}\\

\textcolor{blue}{KCL at Node A:}\\
\begin{equation}
    i_d=i_c+i_s
\end{equation}
\begin{equation}
    u(\frac{V_2-V_a}{2}-V_{th})(\frac{\frac{V_2-V_a}{2}-V_{th}}{R_d})=C(\frac{dV_a}{dt}+\frac{V_a-V_b}{R_s})
\end{equation}\\

\textcolor{blue}{KCL at Node B:}\\
\begin{equation}
    i_s=i_z+i_L+i_{R_L}
\end{equation}
\begin{equation}
    \frac{V_a-V_b}{R_s}=\frac{V_b-V_z}{R_z}u(V_b-V_z)+i_L+\frac{V_b}{R_L}
\end{equation}\\

\textcolor{blue}{KCL at Node C:}\\
\begin{equation}
    i_L=\frac{V_b-V_c}{R_2}
\end{equation}\\

As a result of connecting a resistor and an inductor, the current passing through the resistor Rs can be expressed as follows:\\

\begin{equation}
    \begin{cases}
        \frac{V_a-V_b}{R_s}=i_L+\frac{V_b}{R_L},  V_b<V_z\\
        \frac{V_a-V_b}{R_s}=\frac{V_b-V_z}{R_z}+i_L+\frac{V_b}{R_L},  V_b>V_z
    \end{cases}
\end{equation}\\

Therefore, if $V_b < V_z$, the equation $V_b = V_a - \frac{i_L R_s}{R_s+R_L} R_L$ can be alternatively expressed as:\\

\begin{equation}
    \begin{cases}
        \frac{V_a-V_b}{R_s}=i_L+\frac{V_b}{R_L},  \frac{V_a-i_LR_s}{R_s+R_L}R_L<V_z\\
        \frac{V_a-V_b}{R_s}=\frac{V_b-V_z}{R_z}+i_L+\frac{V_b}{R_L},  \frac{V_a-i_LR_s}{R_s+R_L}>V_z
    \end{cases}
\end{equation}\\

Therefore, the Zener diode is in the cut-off state if $V_a - \frac{i_L R_s}{R_s+R_L} R_L < V_Z$. In this case, the resulting expressions describing the system are:\\


[this below system of equations needs to be check, specifically the = $i_c+i_s$ that eugene has. Should it be $i_s+i_L$ and the value for $i_s $]
\begin{equation}
    \begin{cases}
        u(\frac{V_2-V_a}{2}-V_{th})(\frac{\frac{V_2-V_a}{2}-V_{th}}{R_d})=i_s+i_L\\
        i_s=\frac{V_a-V_b}{R_s}=i_L+\frac{V_b}{R_L}
        i_c=C\frac{dV_a}{dt}\\
        V_c=L\frac{di_L}{dt}\\
        i_s=\frac{V_a-V_b}{R_s}=\frac{V_b-V_z}{R_z}+i_L\\
        V_c=V_b-R_2i_L
    \end{cases}
\end{equation}\\

Therefore, the output voltage, Vb, can be expressed as:\\

\begin{equation}
    V_b=\frac{V-a-i_LR_s}{R_s+R_L}R_L
\end{equation}\\

Therefore, the voltage across the inductor can be described as:\\

\begin{equation}
    V_c=V_b-R_2i_L=\frac{V_a-i_LR_s}{R_s+R_L}R_L-R_2i_L
\end{equation}\\

Moreover, the current flowing through the resistor Rs can be expressed as:\\

\begin{equation}
    i_s=i_L+\frac{V_b}{R_L}=i_L+\frac{V_a-i_LR_s}{R_s+R_L}=\frac{V_a+i_LR_L}{R_s+R_L}
\end{equation}\\

Furthermore, the current flowing through the capacitor can be described as:\\

\begin{equation}
    i_c=i_d-i_s=u(\frac{V_2-V_a}{2}-V_{th})(\frac{\frac{V_2-V_a}{2}-V_{th}}{R_d})-\frac{V_a+i_LR_L}{R_s+R_L}
\end{equation}\\

Consequently, the resulting set of differential equations can be expressed as:\\

\begin{equation}
    \begin{cases}
        \frac{di_L}{dt}=\frac{1}{L}(\frac{V_aR_L-i_L(R_s(R_L+R_2)+R_LR_2)}{(R_L+R_s)})\\
        \frac{dV_a}{dt}=\frac{1}{C}(u(\frac{V_2-V_a}{2}-V_{th})(\frac{\frac{V_2-V_a}{2}-V_{th}}{R_d})-\frac{V_a+i_LR_L}{R_L+R_s}),  \text{if } \frac{V_a-i_LR_s}{R_s+R_L}R_L<V_z
    \end{cases}
\end{equation}\\

In the event that the Zener diode is subjected to reverse bias, and the condition $V_a - \frac{i_L R_s}{R_s+R_L} R_L \geq V_Z$ is satisfied, the system can be described by the following expressions:\\

\begin{equation}
    \begin{cases}
        u(\frac{V_2-V_a}{2}-V_{th})(\frac{\frac{V_2-V_a}{2}-V_{th}}{R_d})=i_s+i_L\\
        i_s=\frac{V_a-V_b}{R_s}=i_L+\frac{V_b}{R_L}
        i_c=C\frac{dV_a}{dt}\\
        V_c=L\frac{di_L}{dt}\\
        V_c=V_b-R_2i_L
    \end{cases}
\end{equation}\\

Therefore, the resulting output voltage, Vb, can be expressed as:\\

\begin{equation}
    V_b=\frac{V_aR_z+V_zR_s-i_LR_sR_z}{R_zR_s+R_LR_z+R_sR_L}R_L
\end{equation}\\

As a result, the voltage across the inductor can be denoted as:\\

\begin{equation}
    V_c=V_b-R_2i_L=\frac{V_aR_z+V_zR_s-i_LR_sR_z}{R_zR_s+R_LR_z+R_sR_L}R_L-R_2i_L
\end{equation}\\

Additionally, the current flowing through resistor Rs can be described as:\\

\begin{equation}
    i_s=\frac{V_a(R_z+R_L)-V_zR_L+i_LR_LR_z}{R_zR_s+R_LR_z+R_sR_L}
\end{equation}\\

Moreover, the current flowing through the capacitor can be expressed as:\\

\begin{equation}
    i_c=i_d-i_s=u(\frac{V_2-V_a}{2}-V_{th})(\frac{\frac{V_2-V_a}{2}-V_{th}}{R_d})-\frac{V_a(R_z+R_L)-V_zR_L+i_LR_zR_L}{R_zR_s+R_LR_z+R_sR_L}
\end{equation}\\

Consequently, the resulting set of differential equations can be summarized as:\\

\begin{equation}
    \begin{cases}
        \frac{di_L}{dt}=\frac{1}{L}(\frac{V_aR_z+V_zR_s-i_LR_sR_z}{R_zR_s+R_LR_z+R_sR_L})\\
        \frac{dV_a}{dt}=\frac{1}{C}(u(\frac{V_2-V_a}{2}-V_{th})(\frac{\frac{V_2-V_a}{2}-V_{th}}{R_d})-\frac{V_a(R_z+R_L)-V_zR_L+i_LR_zR_L}{R_zR_s+R_LR_z+R_sR_L}),  \text{if } \frac{V_a-i_LR_s}{R_s+R_L}R_L \geq V_z
    \end{cases}
\end{equation}\\



