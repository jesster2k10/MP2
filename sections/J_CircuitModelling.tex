\subsection{Circuit Modelling}
Reconfigure the circuit in Figure \ref{fig:powersupply} with the following adjustments:
\begin{itemize}
	\item The original circuit is taken from the LV side of the ideal transformer, thus, the AC voltage source is given by $V_2$ (referring to the ideal transformer model defined in \eqref{eq:ideal_transformer})
	\item The bridge rectifier is taken as two half-rectifiers. One for each half of the cycle (.i.e. half the period of the input sinusoid)
	\item The diodes have been replaced by their linearised models
\end{itemize}

\begin{figure}[H]
	\centering
	
	\begin{circuitikz} [american voltages] \draw
    
    (0,0) to[vsourcesin, l=$V_{2}$] node[label=\textcircled{1}] {} (0,5)
    (0,5) to[battery1, l=$2V_{th}$] (1.9,5)
    (1.5,5) to[R,l=$2R_b$,i=$i_d$, v=$V_d$] (4.5,5)
    (4.5,5) to[nos,o-o] (6,5)
    
    (6,5) node[label=\textcircled{2}] {} to[capacitor, i=$i_c$, l=$C$,v=$V_c$] (6,0)
    
    (6,5) to[R,l=$R_s$,i=$i_s$] (9,5)
    
    (9,5) node[label=\textcircled{3}] {} to[battery1, l=$V_z$] (9,4)
    (9,4) to[R, l=$R_z$, i=$i_z$] (9,2)
    (9,2) to[nos,o-o] (9,0.5)
    (9,0.5) to[short] (9,0)
    
    (9,5) to[short] (12,5)
    (12,5) node[label=\textcircled{4}] {} to[short] (12,4)
    (12,4) to[nos,o-o] (12,3)
    (12,3) to[R, l=$287\Omega$] (12,1)
    (12,1) to[short] (12,0)
    
    (12,5) to[short] (13,5)
    (13,5) to[nos,o-o] (14,5)
    (14,5) to[short] (15,5)
    
    (15,5) node[label=\textcircled{5}] {} to[R,l=$65.7 \Omega$,i=$i_l$] (15,3)
    (15,3) to[L,l=$105.8\text{mH}$] (15,0)
    
    (0,0) to[short] (15,0)
    (7.5,0) node[label=\textcircled{0}] {} node[ground]{} (5,-1);
    
    \end{circuitikz}
	
	\label{circ:multimode}
	\caption{Reconfigured power supply circuit with the multi-mode load}
\end{figure}

\subsubsection{Primary Currents}
The primary currents $i_z$ and $i_d$ are dependant on the unit step function $u(t)$ as earlier discussed.
\paragraph{Diode Current $i_d(t)$} Nodal analysis at \textcircled{2} gives us an expression for $i_d$ in terms of $u(t)$,

\begin{itemize}
	\item The condition for the switch between \textcircled{1} and \textcircled{2} being activated is when
	\begin{equation}
 		\begin{split}
 			V_d > 2V_{th} \\
 			\therefore V_d - 2V_{th} = 0
 		\end{split}
 		\label{eq:ustep_condition}
 	\end{equation}
	\item Thus, the unit step function is unity at $u(V_d > 2V_{th})$
	\item $V_d$ is determined by considering the voltage drop across the two nodes \textcircled{1} and \textcircled{2} $$V_d = e_1 - e_2 - 2V_{th}$$
	\item The current $i_d(t)$ entering the switch is given by $\frac{V_d}{2R_b}$:
	\begin{equation}
		\begin{split}
			i_d(t) &= \frac{e_1 - e_2 - 2V_{th}}{2R_b} \\
				   &= \frac{\frac{e_1 - e_2}{2} - V_{th}}{R_b}	
		\end{split}
	\end{equation}
\end{itemize} 
Thus, to account for the switch we must multiply $i_d(t)$ by the unit step function $u(t)$ that is active when the condition in \eqref{eq:ustep_condition} is met:
\begin{equation}
	i_d(t) = \sbracket{\frac{\frac{e_1 - e_2}{2} - V_{th}}{R_b}}\cdot u\bracket{\frac{e_1 - e_2}{2} - V_{th}}
	\label{eq:diode_current}
\end{equation}

\paragraph{Zener Diode Current $i_z(t)$}
Likewise, the current flowing through the Zener can be determined by considering nodes \textcircled{2} and \textcircled{3}

\begin{itemize}
	\item The condition for the zener diode switch is activated when \begin{equation}
		\begin{split}
			e_3 > V_z \\
			\therefore e_3 - V_z = 0
			\label{eq:zener_unity_condition}
		\end{split}
	\end{equation}
	\item The current $i_z(t)$ entering the switch is given by $\frac{V_z}{R_z}$
	\begin{equation}
		i_z(t) = \frac{e_3 - V_z}{R_z}
	\end{equation}
\end{itemize}
Thus, to account for the switch, multiply $i_z(t)$ by $u(t)$ so that the condition in \eqref{eq:zener_unity_condition} is met.
\begin{equation}
    i_z = \sbracket{\frac{e_3 - V_z}{R_z}}\cdot u\bracket{e_3 - V_z}
    \label{eq:zener_current}
\end{equation}

\paragraph{Node Definitions and Voltage Terms}
Within the subsequent analysis, the node voltages $e_1$ and $e_2$ are frequently utilised.
\begin{itemize}
	\item Here, $e_1$ represents the AC input voltage, $V_2$, while $e_2$ indicates the voltage at node \textcircled{2}, where the capacitor is situated.
	\item The expression $\frac{e_1 - e_2}{2}$ denotes half the voltage difference between the AC input voltage and the node voltage at \textcircled{2}.
	\item The relevance of this difference lies in its critical role in determining the operational mode of the circuit as well as the states of various circuit elements such as the diode and the Zener diode.
\end{itemize}

\pagebreak
\subsection{Modes of Operation}
The circuit operates on several modes determined by the load equivalent circuit model in Figure \ref{fig:multimode_load},
\begin{itemize}
	\item Mode 1: Empty load (.i.e. both switches are turned off)
	\item Mode 2: Inductive load (.i.e. the primary switch is turned off)
	\item Mode 3: Resistive load (.i.e. the secondary switch is turned off) 
	\item Mode 4: Full load (.i.e. all switches are turned on)
\end{itemize}
A set of differential equations describing the operation of mode is determined through nodal analysis

\import{sections/circuit_models/}{NoLoad.tex}
\pagebreak \import{sections/circuit_models/}{InductiveLoad.tex}
\pagebreak \import{sections/circuit_models/}{ResistiveLoad.tex}
\pagebreak \import{sections/circuit_models/}{FullLoad.tex}