\subsection{Circuit Modelling}
Based on the previously discussed assumptions and Figure (1.9), the circuit can be reconfigured as depicted in Figure (2). To elucidate the performance of the circuit in all four load modes, the following equations can be employed:\\

[insert circuit diagram 1.7]\\

\emph{Figure 1.7: Power supply circuit with multimode load $R_L$}

[insert circuit diagram 1.8]\\

\emph{Figure 1.8: Modified circuit with implemented model of circuit elements}\\

The current analysis of the diodes involves utilizing the unit step function, which acts as a switch in the model. When in the on-state, the unit step function is set to 1, and when in the off-state, it is set to 0. This function can be defined as follows:

\begin{equation}
    i_d=u(\frac{V_2-V_a}{2}-V_t)(\frac{\frac{V_2-Va}{2}-V_t}{R_d})
\end{equation}\\

Likewise, the current flowing through the Zener diode:\\

\begin{equation}
    i_z=(\frac{V_b-V_z}{R_z})[u(V_b-V_z)]
\end{equation}\\

\large\textcolor{red}{Mode 1: The load does not include an inductor or resistor (both components are deactivated)}:\\

[insert figure 1.9]\\

\emph{Figure 1.9: Circuit with no load (both devices off)}\\

\textcolor{blue}{KCL at node A:}

\begin{equation}
    i_d=i_c+i_s
    u(\frac{V_2-V_a}{2}-V_t)(\frac{\frac{V_2-Va}{2}-V_t}{R_d})=C(\frac{dV_a}{dt}+\frac{V_a-V_b}{R_s}
\end{equation}\\

where u represents the unit step response.

\textcolor{blue}{KCL at Node B:}
\begin{equation}
    i_s=i_z
    (\frac{V_a-V_b}{R_s}=\frac{V_b-V_z}{R_z}u(V_b-V_z)
\end{equation}\\

Therefore, it is evident that when no load is connected, the current through the Zener can be determined by:

\begin{equation}
    i_z= \begin{cases}
    \frac{V_a-V_b}{R_s}=0,  V_b<V_z\\
    \frac{V_a-V_b}{R_s}=\frac{V_b-V_z}{R_z},  V_b\geq V_z
    \end{cases}
\end{equation}\\

When $V_b$ < $V_z$, $V_a$ is equal to $V_b$, and in this scenario, the current through the Zener can be expressed as:

\begin{equation}
    i_z=\begin{cases}
        0,  V_a<V_z \\
        \frac{V_a-V_b}{R_s}=\frac{V_b-V_z}{R_z},  V_a>V_z
    \end{cases}
\end{equation}

Therefore, if $V_a < V_z$, the Zener diode is in the cut-off region. In this case, the system can be described by the following expressions:\\

\begin{equation}
    \begin{cases}
        i_c=C\frac{dV_a}{dt}\\
        u(\frac{V_2-V_a}{2}-V_{th})(\frac{\frac{V_2-V_a}{2}-V_{th}}{R_d})=i_c
    \end{cases}
\end{equation}\\

Consequently, the differential equation that arises from this is:

\begin{equation}
    \frac{dV_a}{dt}=\frac{1}{C}u(\frac{V_2-V_a}{2}-V_{th})(\frac{\frac{V_2-V_a}{2}-V_{th}}{R_d}),  \emph{if $V_a<V_z$}
\end{equation}

When $V_a \geq V_z$, the Zener diode is in reverse bias. In this situation, the system can be characterized by the following expressions:

\begin{equation}
    \begin{cases}
        i_c=C\frac{dV_a}{dt}\\
        u(\frac{V_2-V_a}{2}-V_{th})(\frac{\frac{V_2-V_a}{2}-V_{th}}{R_d})=i_c+i_s\\
        i_s=\frac{V_a-V_b}{R_s}=\frac{V_b-V_z}{R_z}
    \end{cases}
\end{equation}\\

Thus, it is possible to determine the output voltage, denoted as $V_b$:\\

\begin{equation}
    V_b=\frac{V_aR_z+V_zR_s}{R_s+R_z}
\end{equation}

As a consequence, the following expression describes the current flowing through resistor $R_s$, which can be calculated as:

\begin{equation}
    i_s=\frac{V_a-V_b}{R_s}=\frac{V_a-\frac{V_aR_z+V_zR_s}{R_s+R_z}}{R_s}=\frac{V_a(R_s+R_z)-V_aR_z-V_zR_s}{R_s(R_s+R_z)}=\frac{V_a-V_z}{(R_s+R_z)}
\end{equation}

The current through the capacitor can now be calculated using the following expression:\\

\begin{equation}
    i_c=u(\frac{V_2-V_a}{2}-V_t)(\frac{\frac{V_2-V_a}{2}-V_{th}}{R_d})-\frac{V_a-V_z}{(R_s+R_z)}
\end{equation}


As a result, the differential equation that emerges from this is:

\begin{equation}
    \frac{dV_a}{dt}=\frac{1}{C}\left(u\left(\frac{V_2-V_a}{2}-V_t\right)\left(\frac{\frac{V_2-V_a}{2}-V_{th}}{R_d}\right)-\frac{V_a-V_z}{R_s+R_z}\right), \text{if } V_a \geq V_z
\end{equation}\\

\large\textcolor{red}{\emph{Mode 2: When a resistor is present at the load, the circuit depicted in Figure (1.9) can be simplified as follows:}}\\

[insert figure 2]\\

\emph{Figure 2 illustrates the revised circuit configuration when a resistor load is connected at the output, with one device activated.}\\

\textcolor{blue}{KCL at Node A:}\\

\begin{equation}
    u(\frac{V_2-V_a}{2}-V_{th})(\frac{\frac{V_2-V_a}{2}-V_{th}}{R_d}=C(\frac{dV_a}{dt}+\frac{V_a-V_b}{R_s}
\end{equation}\\

\textcolor{blue}{KCL at Node B:}\\

\begin{equation}
    \frac{V_a-V_b}{R_s}=\frac{V_b-V_z}{R_z}u(V_b-V_z)+\frac{V_b}{R_L}
\end{equation}\\

Therefore, it is evident that when a resistor is connected, the current flowing through resistor Rs can be expressed as:
\begin{equation}
    I_{R_s} = \frac{V_{in} - V_a}{R_s}
\end{equation}


where $V_{in}$ is the input voltage and $V_a$ is the voltage across the load resistor.\\

\begin{equation}
    i_s=\begin{cases}
        \frac{V_a-V_b}{R_s}=\frac{V_b}{R_L},  V_b<V_z\\
        \frac{V_a-V_b}{R_s}=\frac{V_b-V_z}{R_z}+\frac{V_b}{R_L},  V_b \geq V_z
    \end{cases}
\end{equation}\\

When $V_b$ < $V_z$, the voltage $V_b$ can be expressed as $\frac{R_LV_a}{R_s+R_L}$. Therefore, the above expression can be equivalently written as:

\begin{equation}
    \begin{cases}
        \frac{V_a-V_b}{R_s}=\frac{V_b}{R_L}, \frac{R_LV_a}{R_s+R_L}<V_z\\
        \frac{V_a-V_b}{R_s}=\frac{V_b-V_z}{R_z}+\frac{V_b}{R_L}, \frac{R_LV_a}{R_s+R_L} \geq V_z
    \end{cases}
\end{equation}\\

Hence, if $\frac{R_LV_a}{R_s+R_L}<V_z$, the Zener diode is in cut-off mode. In this case, the system can be described by the following expressions:

\begin{equation}
    \begin{cases}
        
    \end{cases}
\end{equation}


