\section{Numerical Analysis}
\subsection{Classic Runge-Kutta Method}

Belonging to a family of iterative methodologies, Runge-Kutta methods, particularly the fourth-order variant (RK4), serve as prominent tools for numerically addressing differential equations. RK4 finds widespread use due to its efficacy in generating solutions to ordinary differential equations (ODEs) that may otherwise be challenging or impractical to solve analytically.

The crux of the Runge-Kutta method lies in using derivative evaluations at several junctures within a designated time step to project an estimate for the function's value at the succeeding step. This is achieved by approximating the Taylor series up to the order of the step size, which is conventionally represented by 'h'.

To illustrate, for a first-order ODE defined as $y' = f(t, y)$, the application of RK4 to progress from time $t$ to $t + h$ would involve the following steps:

\begin{equation}
\begin{aligned}
k_1 &= h \cdot f(t, y), \\
k_2 &= h \cdot f(t + \frac{h}{2}, y + \frac{k_1}{2}), \\
k_3 &= h \cdot f(t + \frac{h}{2}, y + \frac{k_2}{2}), \\
k_4 &= h \cdot f(t + h, y + k_3), \\
y(t + h) &= y(t) + \frac{1}{6} \cdot (k_1 + 2k_2 + 2k_3 + k_4).
\end{aligned}
\end{equation}

In this instance, the calculated 'k' values represent estimations of the function's slope at diverse points within the time step. The method essentially involves crafting a weighted mean of these slopes to derive the next value.\\

Let us define the next step in the solution, $y_{n+1}$, by the equation
\begin{equation}
y_{n+1} = y_n + \frac{h}{6} (k_1 + 2k_2 + 2k_3 + k_4) \quad (3.1)
\end{equation}
Here, $h$ represents the increment in the value of $t$ at each step.

The coefficients $k_i$ denote the slopes at different points in the time step and are defined as follows:
\begin{equation}
f(t, y) = \frac{dy}{dt}\bigg|_{(t, y)} = y'(t, y) \quad (3.2)
\end{equation}
\begin{align*}
k_1 &= f(t_n, y_n), &\text{initial slope using Euler's method},\\
k_2 &= f \left(t_n + \frac{h}{2}, y_n + \frac{k_1}{2}\right), &\text{slope at the midpoint using $y$ and $k_1$},\\
k_3 &= f \left(t_n + \frac{h}{2}, y_n + \frac{k_2}{2}\right), &\text{slope at the midpoint using $y$ and $k_2$},\\
k_4 &= f(t_n + h, y_n + k_3), &\text{slope at the end of the interval using $y$ and $k_3$}.
\end{align*}

These coefficients are defined for $n = 0, 1, 2, 3, \ldots$ . It can be clearly observed that this method bestows higher weights to the midpoint slopes during the computation of their weighted average.

RK4 can be conveniently extended to cater to systems of ordinary differential equations by replacing $y_n$ with the corresponding vector of unknowns.

Four distinctive scenarios are to be considered, each featuring diverse loads connected to the DC rectifier. These will be thoroughly analyzed individually. In the derivation of the equations, an emphasis is placed on optimizing them to incorporate a minimal number of operations. The priority order is addition, multiplication, and division, enabling a reduction in the utilized machine cycles.

\begin{center}
 Establishing Initial Conditions for the Runge-Kutta Algorithm
\end{center}\\

The Runge-Kutta algorithm, like any other method used for numerically solving ordinary differential equations, requires predefined initial conditions to initiate its iteration process. These initial conditions serve as the springboard for the subsequent calculations. In the context of an electrical circuit, these conditions are established based on the state of the system at the start time, typically denoted as t = 0s.

For a circuit that has been left in a stationary state before power supply is connected, we can infer that all transient effects have subsided and the system is in a state of equilibrium. As such, it's reasonable to assume that the capacitor in the circuit is completely discharged, indicating a zero voltage across it. Similarly, no currents are flowing through the circuit elements as no potential difference exists to drive them. These constitute the initial conditions at the outset for the Runge-Kutta algorithm's iterations.

\begin{center}
     No-Load Behavior and Single Reactive Element Analysis
\end{center}\\

Under a no-load condition, where the output is an open circuit, the system exhibits a distinct behavior. In such cases, it is evident that the system involves a single reactive element, specifically a capacitor. Consequently, the complexity of the system is reduced, and it can be described by a single ordinary differential equation (ODE). This ODE captures the relationship between the variables \(y\) and its derivative \(y'\), serving as a fundamental equation governing the system's behavior in the absence of a load. By analyzing and solving this ODE, we can gain insights into the dynamics and characteristics of the system when it operates without any external load connected to it.\\

Consequently, we can express the resulting equations for \(y\) and \(y'\) (the derivatives of \(y\)) as follows:\\

\begin{equation}
    y=[V_a]    
\end{equation}\\

\begin{equation}
    y'=[\frac{dV_a}{dt}]=[\frac{i_c}{C}]
\end{equation}\\

The expression for $y'$ of the Zener diode in the reverse-biased state can be obtained by utilizing Equation (1.65):

\begin{equation}
    \frac{d(e_2)}{dt}=\frac{1}{C}\sbracket{\bracket{\frac{\frac{e_1 - e_2}{2} - V_{th}}{R_b}}\cdot u\bracket{\frac{e_1 - e_2}{2} - V_{th}}-\frac{e_2-V_z}{R_s+R_z}} \hspace{0.1cm} \text{for} \hspace{0.1cm} e_2 \geq V_z
\end{equation}\\

\begin{equation}
    => y'=\frac{1}{C}[(u(\frac{V_2-e_2}{2}-V_{th})(\frac{\frac{V_2-e_2}{2}-V_{th}}{R_b})-\frac{V_a-V_z}{R_s+R_z}]
\end{equation}\\

The expression for $y'$ of the Zener diode in the cut-off state can be obtained by utilizing Equation (1.58):

\begin{equation}
     \frac{d(e_2)}{dt}=\frac{1}{C} \sbracket{\bracket{\frac{\frac{e_1 - e_2}{2} - V_{th}}{R_b}} \cdot u\bracket{\frac{e_1-e_2}{2}-V_{th}}} \hspace{0.5cm} \text{for} \hspace{0.1cm} e_2 < V_z
    \label{eq:de2dt_reverse_bias}
\end{equation}\\

\begin{equation}
    =>y'=\frac{1}{C}[u(\frac{V_2-V_a}{2}-V_t)\frac{\frac{V_2-V_a}{2}-V_t}{R_d}]
\end{equation}\\

The voltage of the capacitor for the next iteration in the Runge-Kutta algorithm can be calculated using Equations (2.7) and (2.9), taking into account the state of the Zener diode. Additional parameters can be determined by considering the input voltages $V_2$ and Va at each iteration.

To evaluate the state of the Zener diode, the voltage across it is examined at the beginning of each iteration. If this voltage is equal to or exceeds the threshold voltage $V_z$ of the Zener diode, it is considered to be in a reverse bias state. Conversely, if the voltage is below this threshold, the diode is in a cut-off state. The expression to calculate the voltage across the Zener diode at any given time can be derived from Equation (2.3).

\begin{equation}
    V_b=\begin{cases}
        V_a, V_b \leq V_z\\
        \frac{V_aR_z+V_zR_s}{R_s+R_z}, V_b>V_z
    \end{cases}
\end{equation}\\

The equation (2) can be utilized to determine the current flowing through the Zener diode:\\

\begin{equation}
    i_z=\frac{V_b-V_z}{R_z}u(V_b-V_z)
\end{equation}\\

In this scenario, the current flowing through the resistor Rs is equal to the current passing through the Zener diode. However, an alternative approach to determine this current through the resistor can also be derived using:\\

\begin{equation}
    i_s=\frac{V_a-V_b}{R_s}
\end{equation}\\

The Equation (1.9) provides the necessary information to determine the current through the diodes:\\

\begin{equation}
    i_d=u(\frac{V_2-V_a}{2}-V_t)(\frac{\frac{V_2-V_a}{2}-V_t}{R_d})
\end{equation}\\

The current through the capacitor can be expressed as follows:\\

\begin{equation}
    i_c=i_s-i_d
\end{equation}\\

\textcolor{blue}{Resistive Load}\\

When considering a resistive load, the characteristics and behavior of the system can be further elaborated. Unlike reactive elements, such as capacitors or inductors, which introduce phase shifts and complex dynamics, resistive loads primarily consist of resistive elements like resistors. These resistive elements do not cause significant phase differences and primarily dissipate electrical energy as heat.

In the specific scenario of a resistive load, the system's equations and calculations revolve around the properties of resistive elements. These elements allow for simpler and more direct mathematical representations, enabling easier analysis and understanding of the system's behavior. By incorporating the appropriate equations and considering the characteristics of resistive loads, we can gain valuable insights into the system's performance, stability, and overall dynamics under these conditions.

In scenarios where it is evident that only one reactive element, specifically a capacitor, is present, the system can be effectively described by a single ordinary differential equation (ODE). Consequently, the resulting equations for $y$ and $y'$ can be expressed as follows:

\begin{equation}
    y=[V_a]
\end{equation}\\

\begin{equation}
    y'=[\frac{dV_a}{dt}]=[\frac{i_c}{C}]
\end{equation}\\

The equation (1.78) provides the derived expression for $y'$ of the Zener diode in its reverse-biased state.\\

\begin{equation}
    \frac{de_2}{dt}=\frac{1}{C}\left(u\left(\frac{V_2-e_2}{2}-V_{th}\right)\left(\frac{\frac{V_2-e_2}{2}-V_{th}}{R_b}\right)-\frac{e_2(R_L+R_z)-R_LV_Z}{R_zR_L+R_s(R_z+R_L)}\right), \text{if } \frac{R_Le_2}{R_s+R_L} \geq V_z
\end{equation}\\

\begin{equation}
    =>y'=\frac{1}{C}[u(\frac{V_2-V_a}{2}-V_t)(\frac{\frac{V_2-V_a}{2}}{R_d}-\frac{e_2(R_L+R_z)-R_LV_Z}{R_zR_L+R_s(R_z+R_L)})]
\end{equation}

The equation (1.65) yields the determined value for $y'$ of the Zener diode when it is in the cut-off state:\\

\begin{equation}
    \frac{d(e_2)}{dt}=\frac{1}{C}\sbracket{\bracket{\frac{\frac{e_1 - e_2}{2} - V_{th}}{R_b}}\cdot u\bracket{\frac{e_1 - e_2}{2} - V_{th}}-\frac{e_2-V_z}{R_s+R_z}} \hspace{0.1cm} \text{for} \hspace{0.1cm} e_2 \geq V_z
\end{equation}

\begin{equation}
    y'=\frac{1}{C}[u(\frac{V_2-V_a}{2}-V_t)(\frac{\frac{V_2-V_a}{2}-V_t}{R_d}-\frac{V_a}{(R_s+R_L)}]
\end{equation}\\

The equations (3.4.3) and (3.4.4) provide a means to calculate the voltage at the capacitor for the next iteration using the Runge-Kutta method, taking into account the state of the Zener diode. Additional parameters can be determined by utilizing the input voltages V2 and Va at each iteration.

The condition for the Zener modes remains the same as described previously. In this particular case, the expression for Vb can be derived using Equation (2.8.2).\\

\begin{equation}
    V_b=\begin{cases}
        \frac{\frac{V_a}{R_s}}{\frac{1}{R_s+\frac{1}{R_L}}}, V_b \leq V_z\\
        \frac{\frac{V_a}{R_s}+\frac{V_z}{R_z}}{\frac{1}{R_s}+\frac{1}{R_L}+\frac{1}{R_z}}, V_b>V_z
    \end{cases}=\begin{cases}
        \frac{R_LV_a}{R_s+R_L}, V_b \leq V_z\\
        \frac{R_zV_a+R_sV_z}{\frac{R_zR_s}{R_L}+R_z+R_s}, V_b>V_z
    \end{cases}
\end{equation}\\

The expression for the current flowing through the load resistance $R_L$ can be expressed as:\\

\begin{equation}
    I_{R_L}=\frac{V_b}{R_L}
\end{equation}\\

All other parameters can be determined using the previously derived equations.

\textcolor{blue}{Inductive Load}\\

 The circuit under consideration involves two reactive elements, namely an inductor and a capacitor. This leads to a more complex system that can be described by a system of ordinary differential equations (ODEs). Consequently, the resulting equations for the vectors $y$ and $y'$, which represent the variables and their derivatives, can be formulated to capture the behavior of the circuit. Taking into account the inductive and capacitive elements, the analysis of this system requires an understanding of the dynamics, interactions, and interplay between these reactive components. By solving the system of ODEs, we can gain insights into the voltage and current relationships, energy transfers, and the overall behavior of the circuit under the influence of both the inductor and the capacitor.\\

In this scenario, the circuit comprises both an inductor and a capacitor as reactive elements. As a result, it is represented by a system of ODEs, where the equations involving the vectors $y$ and $y'$ describe its behavior. Solving these equations provides insights into the circuit's voltage-current relationships and energy dynamics.\\

\begin{equation}
    y=[\frac{V_a}{I_L}]
\end{equation}\\

\begin{equation}
    y'=[\begin{cases}
        \frac{dV_a}{dt}\\
        \frac{dI_L}{dt}
    \end{cases}]=[\begin{cases}
        \frac{i_c}{C}\\
        \frac{V_c}{L}
    \end{cases}]
\end{equation}\\

The equation (1.94) yields the derived expression for $y'$ of the Zener diode in its reverse-biased state:\\

\begin{equation}
    \begin{cases}
        \frac{di_L}{dt}=\frac{1}{L}(\frac{R_ze_2+R_sV_z-i_L(R_s(R_z+R_2)+R_2R_z)}{(R_z+R_s)})\\
        \frac{de_2}{dt}=\frac{1}{C}(u(\frac{V_2-e_2}{2}-V_{th})(\frac{V_2-e_2}{2}-V_{th})(\frac{\frac{V_2-e_2}{2}-V_{th}}{R_b})-\frac{e_2-V_z+i_LR_z}{R_z+R_s}),  \text{if } e_2-R_si_L \geq V_z
    \end{cases}
\end{equation}\\

\begin{equation}
    y'=[\begin{cases}
        \frac{1}{C}(u(\frac{V_2-V_a}{2}-V_t)(\frac{\frac{V_2-V_a}{2}-V_t}{R_d}-\frac{V_a-V_z+i_LR_z}{(R_z+R_s)})\\
        \frac{1}{L}(\frac{R_zV_a+R_sV_z-i_L(R_s(R_z+R_2)+R_2R_z)}{R_z+R_s})
    \end{cases}]
\end{equation}\\

The equation (1.89) provides the determined expression for $y'$ of the Zener diode when it is in the cut-off state.\\

\begin{equation}
    \begin{cases}
        \frac{di_L}{dt}=\frac{1}{L}(e_2-(R_s+R_2)i_L)\\
        \frac{de_2}{dt}=\frac{1}{C}\left(u\left(\frac{V_2-e_2}{2}-V_{th}\right)\left(\frac{V_2-e_2}{2}-V_{th}\right)\left(\frac{\frac{V_2-e_2}{2}-V_{th}}{R_b}\right)-i_L\right), \text{if } e_2-R_si_L<V_z
    \end{cases}
\end{equation}\\

\begin{equation}
    =>y'=[\begin{cases}
        \frac{1}{C}(u(\frac{V_2-V_a}{2}-V_t)(\frac{\frac{V_2-V_a}{2}-V_t}{R_d}-I_L)\\
        \frac{1}{L}(V_a-(R_s+R_2)I_L)
    \end{cases}]
\end{equation}\\

Equations $(3.5.3)$ and $(3.5.4)$ provide a means to determine the voltage across the capacitor and the current flowing through the inductor at the next iteration within the Runge-Kutta algorithm. These calculations depend on the state of the Zener diode. Other parameters can be obtained by considering the input voltage $V_2$, $V_a$, and $I_L$ at each iteration.

The conditions for the Zener modes remain consistent with what has been previously discussed. In this specific scenario, the value of $V_b$ can be derived from Equation $(2.14)$.\\


[need to change this equation to equation 3.5.5 in eugenes]
\begin{equation}
    V_b=\begin{cases}
        V_a-R_sI_L, V_b \leq V_z\\
        \frac{\frac{V_a}{R_s}-I_L+\frac{V_z}{R_z}}{\frac{1}{R_s+\frac{1}{R_z}}}, V_b>V_z
    \end{cases}=\begin{cases}
        V_a-R_sI_L, V_b \leq V_z\\
        \frac{R_z(V_a-I_LR_s)+V_zR_s}{R_s+R_z}, V_b>V_z
    \end{cases} 
\end{equation}\\

The expression for the current flowing through resistor R2 is as follows:\\

\begin{equation}
    I_{R_2}=I_L
\end{equation}\\

The voltage across the inductor Vc can be expressed as:\\

\begin{equation}
    V_c=V_b-R_2I_L
\end{equation}\\

The remaining parameters can be determined utilizing previously derived equations.\\

\textcolor{blue}{Full Load}

The circuit being discussed involves two reactive elements, specifically an inductor and a capacitor. As a result, the behavior of the circuit can be accurately described by a system of ordinary differential equations (ODEs). These equations, involving the vectors $y$ and $y'$, capture the dynamics and interactions between the inductor and capacitor within the circuit.

Now, shifting our focus to the concept of full load, it is important to recognize that full load conditions can significantly impact the performance of a circuit or system. Under full load, the system operates at its maximum or rated capacity, experiencing the highest current or power demand. This state is crucial for evaluating the reliability, efficiency, and functionality of the circuit, ensuring that it can handle the load without compromising its performance.

When examining the behavior of a circuit under full load, considerations such as power consumption, heat dissipation, and voltage stability become paramount. Engineers and designers must carefully size and design the circuit to meet the demands of full load conditions, ensuring safe and efficient operation.

Understanding the behavior of a circuit with reactive elements, as described in your friend's message, allows us to evaluate its performance under full load accurately. By considering the derived equations for $y$ and $y'$ within the context of full load, we can assess the system's capability to handle maximum demands and determine its overall reliability and efficiency under these challenging operating conditions.

\begin{equation}
    y=[V_a\\
    I_L]
\end{equation}

\begin{equation}
    y'=[\frac{dV_a}{dt}\\
    \frac{dI_L}{dt}]=[\frac{i_c}{C}\\
    \frac{V_c}{L}]\\
\end{equation}


$y'$ for Zener in reverse-biased state has been found from the Equations (1.93, 1.94, 1.95, 1.96):\\

\begin{equation}
    y'=[\frac{1}{C}(u(\frac{V_2-e_2}{2}-V_{th})(\frac{V_2-e_2}{2}-V_{th})(\frac{\frac{V_2-e_2}{2}-V_{th}}{R_b})-\frac{V_a(R_z+R_L)+R_L(i_LR_z-V_z)}{(R_s(R_z+R_L)+R_LR_z)}\\
    \frac{1}{L}(R_L\frac{V_aR_z+(V_z-i_LR_z)R_s}{(R_s(R_z+R_L)+R_LR_z)}-R_2i_L)]
\end{equation}\\

The equation (2.21.6) provides the derived expression for $y'$ of the Zener diode when it is in the cut-off state.\\

\begin{equation}
    y'=[\frac{1}{C}(u((\frac{V_2-e_2}{2}-V_{th})(\frac{\frac{V_2-e_2}{2}-V_{th}}{R_b})-\frac{V_a+i_LR_L}{R_s+R_L}\\
    \frac{1}{L}(\frac{-i_L(R_s(R_L+R_2)+R_LR_2)+V_aR_L}{R_s+R_L})]
\end{equation}\\

By utilizing Equations (3.6.3) and (3.6.4), it is possible to determine the voltage across the capacitor and the current flowing through the inductor at the subsequent iteration in the Runge-Kutta algorithm. These calculations are contingent upon the state of the Zener diode. Moreover, by considering the input voltage V2, Va, and IL at each iteration, it becomes feasible to ascertain other parameters.\\

\begin{equation}
    V_b=\begin{cases}
        \frac{\frac{V_a}{R_s}-I_L}{\frac{1}{R_s}+\frac{1}{R_L}}, V_b \leq V_z\\
        \frac{\frac{V_a}{R_s}-I_L+\frac{V_z}{R_z}}{\frac{1}{R_s}+\frac{1}{R_L}+\frac{1}{R_z}}, V_b>V_z
    \end{cases}=\begin{cases}
        R_L\frac{V_a-I_LR_s}{R_L+R_s}, V_b \leq V_z\\
        \frac{R_zV_a-I_LR_sR_z+V_zR_s}{R_s+\frac{R_sR_z}{R_L}+R_z}, V_b>V_z
    \end{cases}
\end{equation}\\

The remaining parameters can be determined by employing previously derived equations.\\

\textcolor{blue}{Switching in Motion}\\

In the realm of switching in motion, as your friend highlighted, the ability to connect or disconnect loads using switches while power is applied to the circuit is a crucial aspect. When dealing with resistive loads, the previously derived equations remain accurate, assuming an ideal resistor with no dynamic response to changes in voltage or current.

For connecting an inductive load, it becomes essential to ensure that the current through the inductor is maintained at $0A$. Under this condition, the aforementioned equations remain valid. However, the model described above does not encompass the scenario when an inductive load is disconnected. Previously, it was assumed that the switch is ideal, leading to instantaneous disconnection as the time interval ($\Delta t$) approaches zero ($\Delta t \to 0s$). However, in practical situations, when the switch is flipped, the current through the inductor will not instantaneously reduce to zero. As a result, the voltage across the inductor can be expressed using the equation [4]:

This highlights the need to consider the dynamics of inductive loads during switching in motion, as the instantaneous reduction of current to zero may not be achieved. Proper analysis and understanding of these dynamics are essential for accurate modeling and effective control of systems involving inductive loads.

\begin{equation}
    V_L=L\frac{di_L}{dt} \approx \lim_{\Delta t \to 0} L\frac{\Delta i_L}{\Delta t}=-Li_L \lim_{\Delta t \to 0} \frac{1}{\Delta t}
\end{equation}\\

Consequently, the solution tends to negative infinity (- $\infty$), which lacks physical significance. In actuality, the voltage will experience a spike; however, within any switch, an electrical breakdown will transpire, preventing the current from instantaneously reducing to zero. It would be more reasonable to model the voltage across the diode in the subsequent cycle after the switch is turned off as:\\

\begin{equation}
    V^{n+1}_L = -\frac{Li^{n}_L}{h}
\end{equation}\\

As a result, the current in the subsequent cycle is $i_L^{n+1} = 0$ (A). Consequently, in all subsequent cycles, both the voltage and current will be zero. Since the resistor is not a reactive element, the voltage across it will also be zero after the switch is flipped.

It is important to note that this model cannot accurately predict the actual voltage across the inductor. Instead, it serves to indicate a surge in the inductor voltage after it is disconnected from the power supply.\\

\Large\textcolor{blue}{Implementation of Numerical Algorithm for Circuit Analysis}

The implementation of the numerical algorithm for the circuit analysis involved the development of a MATLAB function m-file, utilizing the derived equations as discussed earlier. This m-file takes various inputs, including the current voltage across the Zener diode, capacitor, and the current flowing through the inductor. Additionally, the input voltage at the present, intermediate, and next iteration is provided to determine the slopes at different time intervals. The timestep value, mode of operation, and relevant parameters of the circuit are also considered as inputs, as they may vary.

The MATLAB function m-file then performs the necessary calculations and computations to determine all the currents and voltages in the circuit for the next iteration. By utilizing the numerical algorithm and considering the changing inputs, the function provides a comprehensive analysis of the circuit dynamics, facilitating the evaluation of various parameters and their impact on the circuit's behavior.

This implementation enables efficient and accurate simulation and analysis of the circuit, allowing for insights into its performance under different operating conditions. It serves as a valuable tool for engineers and researchers in studying and optimizing circuit designs, understanding the effects of parameter variations, and making informed decisions for circuit improvement.\\

[Code Section 1]\\

\emph{Figure 3.1: MATLAB Function M-File for Computing Voltages and Currents in the Circuit at Time n+1}\\

\Large\textcolor{blue}{\textbf{Refining Timestep Selection for Accurate Circuit Analysis}}


To determine an appropriate timestep for the numerical algorithm, the input to the circuit, in this case, is a mains signal with a frequency of 50 Hz. Through a process of trial and error, it was found that a timestep of 20 μs, resulting in 1000 cycles per second, is sufficient for accurate simulations. However, it is sensible to verify this choice by comparing the results with a smaller timestep, such as 10 μs, which is half the size of 20 μs. If the difference between the resulting waveforms produced by the RungeKuttaCircuit using both timesteps is negligible, it indicates that the chosen timestep size is acceptable.

To conduct this comparison, the voltage across the capacitor (Va) and the voltage across the load (Vb) can be analyzed during the first second after the power is supplied. This analysis should be performed for all four modes of operation, taking into account the nominal values of the resistor (Rs) and capacitor (C).

By evaluating the waveforms and assessing the level of deviation between the two timesteps, engineers and researchers can confidently determine if the timestep of 20 μs is appropriate for accurately capturing the circuit behavior. This verification process helps ensure that the chosen timestep strikes a balance between computational efficiency and accuracy, resulting in reliable simulations for further analysis and evaluation.\\

[Graph for Circuit Parameters versus time without load]\\

\emph{Figure 3.2.1: Voltage across the load and capacitor, and current through the Zener diode under no-load conditions}\\


















\subsection{Numerical Algorithim}
\subsection{Timestep Scale}
\subsection{Maximial Diode Current}