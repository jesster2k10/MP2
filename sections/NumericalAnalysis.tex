\section{Numerical Analysis}
Differential equations describing the circuit under all four modes of operation have been derived:
\begin{itemize}
    \item \eqref{eq:noLoad_diffEq1} \eqref{eq:noLoad_diffEq2} describe the circuit under \textbf{no load}
    \item \eqref{eq:resistiveLoad_DiffEq1} \eqref{eq:resistiveLoad_DiffEq2}  describe the circuit under a \textbf{resistive load}
    \item \eqref{eq:inductiveLoad_DiffEq1} \eqref{eq:inductiveLoad_DiffEq2} \eqref{eq:inductiveLoad_DiffEq3} \eqref{eq:inductiveLoad_DiffEq4} describe the circuit under an \textbf{inductive load}
    \item \eqref{eq:fullLoad_DiffEq1} \eqref{eq:fullLoad_DiffEq2} \eqref{eq:fullLoad_DiffEq3} \eqref{eq:fullLoad_DiffEq4} describe the circuit under a \textbf{full load}
\end{itemize}

The 4th order variation of the Runge-Kutta (RK4) method is used to solve these differential equations

\subsection{Runge-Kutta Method}
\subsubsection{Overview}
Runge-Kutta is a numerical method used to solve initial value ordinary differential equations
\begin{itemize}
    \item It approximates the solution by dividing the interval into smaller steps and iteratively computing the solution at each step \citep{rungeKutta}
    \item The fourth order variation of Runge-Kutta (RK4) is used in this numerical analysis as it offers a good balance between accuracy and computation cost
\end{itemize}

The core mechanism of the Runge-Kutta method encompasses the evaluation of derivatives at multiple instances within a specific time interval, $h$. Through this process, an estimation of the function's subsequent value is projected. This strategy achieves its efficacy by implementing an approximation to the Taylor series, truncated according to the designated step size, $h$ \citep{rungeKutta}

\subsubsection{Analytical Breakdown}
For a first order ordinary differential equation defined as $\frac{dy}{dt} = f(t, y)$ the application of RK4 to progress from time $t$ to $t + h$ would involve the following steps
\begin{equation}
\begin{aligned}
k_1 &= hf(t, y), \\
k_2 &= hf(t + \frac{h}{2}, y + \frac{k_1}{2}), \\
k_3 &= hf(t + \frac{h}{2}, y + \frac{k_2}{2}), \\
k_4 &= hf(t + h, y + k_3), \\
y(t + h) &= y(t) + \frac{1}{6}(k_1 + 2k_2 + 2k_3 + k_4)
\end{aligned}
\end{equation}
RK4 essentially crafts a weighted mean of these slopes to derive the next value at time $t + h$. Define the next step in the solution, $y_{n+1}$, by the equation
\begin{equation}
y_{n+1} = y_n + \frac{h}{6} (k_1 + 2k_2 + 2k_3 + k_4)
\end{equation}
Here, $h$ represents the increment in the value of $t$ at each step.

The coefficients $k_n$ denote the slopes at different points in the time step and are defined as follows:
\begin{equation}
f(t, y) = \frac{dy}{dt}\bigg|_{(t, y)} = y'(t, y)
\end{equation}
\begin{align*}
k_1 &= f(t_n, y_n), &\text{initial slope using Euler's method},\\
k_2 &= f \left(t_n + \frac{h}{2}, y_n + \frac{k_1}{2}\right), &\text{slope at the midpoint using $y$ and $k_1$},\\
k_3 &= f \left(t_n + \frac{h}{2}, y_n + \frac{k_2}{2}\right), &\text{slope at the midpoint using $y$ and $k_2$},\\
k_4 &= f(t_n + h, y_n + k_3), &\text{slope at the end of the interval using $y$ and $k_3$}.
\end{align*}
Where $k_n$ is defined for $1 \leq n \leq \infty$. The method assigns higher weights to midpoint slopes, and it extends to systems of ordinary differential equations. 

\subsubsection{Modes of Operation}
The initial condition stated in section \ref{determinationOfMaximimumCurrent} holds true for all modes of operation \eqref{initialCondition} 

\import{sections/numerical/}{NoLoad.tex}
\pagebreak \import{sections/numerical/}{InductiveLoad.tex}
\import{sections/numerical/}{ResistiveLoad.tex}
\pagebreak\import{sections/numerical/}{FullLoad.tex}

\subsubsection{Progression Step}
To demonstrate the functionality of the RK4 method, let's consider a scenario of no load operation. In this case, the zener diode is in reverse breakdown. The RK4 progression at time $t=1ms$ with $n=20$ and a time step of $h=0.05ms$ is governed by the following equations
\begin{equation}
	\begin{split}
		k_1 &= \frac{0.05}{C}\sbracket{\bracket{\frac{\frac{e_1 - y}{2} - V_{th}}{R_b}}\cdot u\bracket{\frac{e_1 - y}{2} - V_{th}}-\frac{y-V_z}{R_s+R_z}} \\
	k_2 &= \frac{0.05}{C}\sbracket{\bracket{\frac{\frac{e_1 - (y+\frac{k_1}{2})}{2} - V_{th}}{R_b}}\cdot u\bracket{\frac{e_1 - (y+\frac{k_1}{2})}{2} - V_{th}}-\frac{(y+\frac{k_1}{2})-V_z}{R_s+R_z}} \\
	k_3 &= \frac{0.05}{C}\sbracket{\bracket{\frac{\frac{e_1 - (y+\frac{k_2}{2})}{2} - V_{th}}{R_b}}\cdot u\bracket{\frac{e_1 - (y+\frac{k_2}{2})}{2} - V_{th}}-\frac{(y+\frac{k_2}{2})-V_z}{R_s+R_z}} \\
	k_4 &= \frac{0.05}{C}\sbracket{\bracket{\frac{\frac{e_1 - (y+k_3)}{2} - V_{th}}{R_b}}\cdot u\bracket{\frac{e_1 - (y+k_3)}{2} - V_{th}}-\frac{(y+k_3)-V_z}{R_s+R_z}}
	\end{split}
\end{equation}
Hence, solve for the capacitor voltage $e_2$ at time $t=1.05ms$ with $n=21$
\begin{equation}
	e_2_{(n=21)} = e_2_{(n=20)} + \frac{0.05}{6}(k_1+2k_2+2k_3+k_4)
\end{equation}
This process is iteratively repeated to solve for $y_n$ across a range of time indices in the simulation.

