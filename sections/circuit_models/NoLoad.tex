\subsubsection{Empty Load}
Both switches are off, meaning that there is no load on the power supply, hence it is called 'Empty Load'. The equivalent circuit given an empty load is simply an open circuit across the zener diode.

\begin{figure}[H]
	\centering
	
	\begin{circuitikz} [american voltages] \draw
    
    (0,0) to[vsourcesin, l=$V_{2}$] node[label=\textcircled{1}] {} (0,5)
    (0,5) to[battery1, l=$2V_{th}$] (1.9,5)
    (1.5,5) to[R,l=$2R_b$,i=$i_d$, v=$V_d$] (4.5,5)
    (4.5,5) to[nos,o-o] (6,5)
    
    (6,5) node[label=\textcircled{2}] {} to[capacitor, i=$i_c$, l=$C$,v=$V_c$] (6,0)
    
    (6,5) to[R,l=$R_s$,i=$i_s$] (9,5)
    
    (9,5) node[label=\textcircled{3}] {} to[battery1, l=$V_z$] (9,4)
    (9,4) to[R, l=$R_z$, i=$i_z$] (9,2)
    (9,2) to[nos,o-o] (9,0.5)
    (9,0.5) to[short] (9,0)
    
    (12,5) to[open,v=$e_3$] (12,0)
    
    (9,5) to[short,-o] (12,5)
    (0,0) to[short,-o] (12,0)
    (6,0) node[ground]{} (5,-1);
    
    \end{circuitikz}
	
	\label{circ:empty_load}
	\caption{Power supply circuit with an empty load}
\end{figure}

\paragraph{Nodal Anaylsis}
\subparagraph{Node \textcircled{2}}
Applying KCL to determine an expression for $i_d(t)$ at node \textcircled{2}
\begin{equation}
    \begin{split}
    	i_d(t) = i_c(t) + i_s
    \end{split}
\end{equation}
Apply the primary equation for $i_d(t)$ \eqref{eq:diode_current}, the ideal capacitor model \eqref{eq:ideal_capacitor} and Ohm's law,
\begin{equation}
	\sbracket{\frac{\frac{e_1 - e_2}{2} - V_{th}}{R_b}}\cdot u\bracket{\frac{e_1 - e_2}{2} - V_{th}} = C \frac{d(e_2)}{dt} + \frac{e_2 - e_3}{R_s}
	\label{eq:mode1_node2}
\end{equation}

\subparagraph{Node \textcircled{3}}
Applying KCL to determine an expression for $i_s$ at node \textcircled{3}
\begin{equation}
    \begin{split}
        i_s &= i_z(t)
    \end{split}
\end{equation}
Here, \(i_z(t)\) depends on whether \(e_3 < V_z\) or \(e_3 \geq V_z\). Therefore, we have

\begin{equation}
    \frac{e_2 - e_3}{R_s} = \sbracket{\frac{e_3 - V_z}{R_z}}\cdot u\bracket{e_3 - V_z}
    \label{eq:mode1_node3}
\end{equation}

\paragraph{Differential Equations}
Two distinct cases emerge from the given conditions, each leading to a unique differential equation. These cases are determined by the relationship between \(e_3\) and \(V_z\).\\

In the first case where \(e_3 < V_z\), the Zener diode remains in \textbf{cut off region}, and the circuit is described by the following differential equation:
\begin{equation}
    \frac{d(e_2)}{dt}=\frac{1}{C} \sbracket{\bracket{\frac{\frac{e_1 - e_2}{2} - V_{th}}{R_b}} \cdot u\bracket{\frac{e_1-e_2}{2}-V_{th}}} \hspace{0.5cm} \text{for} \hspace{0.1cm} e_3 < V_z
\end{equation}

In the second case where \(e_3 \geq V_z\), the Zener diode is \textbf{in reverse breakdown}. The differential equation characterising the circuit in this state is:
\begin{equation}
   \frac{d(e_2)}{dt}=\frac{1}{C}\sbracket{\bracket{\frac{\frac{e_1 - e_2}{2} - V_{th}}{R_b}}\cdot u\bracket{\frac{e_1 - e_2}{2} - V_{th}}-\frac{e_2-V_z}{R_s+R_z}} \hspace{0.1cm} \text{for} \hspace{0.1cm} e_3 \geq V_z
\end{equation}

\subparagraph{Cut off}
During the cut off state of the Zener diode, the voltage $e_3$ across the diode may fall below the Zener voltage $V_z$. The current $i_z$ flowing through the Zener diode under the no-load condition can be expressed as a function of $e_2$ and $e_3$ as given by \eqref{eq:mode1_node3},

\begin{equation}
	i_z = \begin{cases}
		0, & \text{if } e_3 < V_z \\
		\frac{e_2 - e_3}{R_s}=\frac{e_3 - V_z}{R_z}, & \text{if } e_3 \geq V_z
	\end{cases}
    \label{eq:iz_reverse_bias}
\end{equation}

When $e_3 < V_z$, the voltage $e_2$ is equivalent to $e_3$ rendering the current through the Zener diode, $i_z$, to be zero. Thus, the Zener current can be rewritten as:
\begin{equation}
    i_z=\begin{cases}
        0,  & \text{if } e_2 < V_z \\
        \frac{e_2-e_3}{R_s}=\frac{e_3-V_z}{R_z}, & \text{if } e_2 \geq V_z
    \end{cases}
    \label{eq:iz_condensed}
\end{equation}

In the first case of \eqref{eq:iz_condensed}, the Zener is in reverse cut off. The system's state can be described by:
\begin{equation}
    \begin{split}
        i_c(t) &= C\frac{d(e_2)}{dt} \\
        i_c(t) &= \sbracket{\frac{\frac{e_1 - e_2}{2} - V_{th}}{R_b}}\cdot u\bracket{\frac{e_1 - e_2}{2} - V_{th}}
    \end{split}
    \label{eq:ic_reverse_bias}
\end{equation}

\pagebreak
The above equations are a direct result of applying Kirchhoff's Current Law at node \textcircled{2} and considering the capacitor's current-voltage relationship. Consequently, the arising differential equation is given by:
\begin{equation}
     \frac{d(e_2)}{dt}=\frac{1}{C} \sbracket{\bracket{\frac{\frac{e_1 - e_2}{2} - V_{th}}{R_b}} \cdot u\bracket{\frac{e_1-e_2}{2}-V_{th}}} \hspace{0.5cm} \text{for} \hspace{0.1cm} e_2 < V_z
    \label{eq:noLoad_diffEq1}
\end{equation}

\eqref{eq:noLoad_diffEq1} provides the rate of change of the voltage $e_2$ across the capacitor for $e_2 < V_z$. The unit step function, denoted by $u(\cdot)$, ensures that the right-hand side is zero whenever the condition inside the brackets is less than zero. This behaviour aligns with the physical scenario of the diode being in the off-state under reverse cut off conditions when $e_2 < V_z$.

\subparagraph{Reverse Breakdown}
In the condition where $e_3 \geq V_z$, the Zener diode transitions to its reverse breakdown region. The circuit characteristics during this state are governed by the equations presented below.
\begin{align}
    i_c(t) &= C\frac{d(e_2)}{dt}\\
    i_c(t) + i_s &= \sbracket{\frac{\frac{e_1 - e_2}{2} - V_{th}}{R_b}}\cdot u\bracket{\frac{e_1 - e_2}{2} - V_{th}}\\
    i_s &= \frac{e_2-e_3}{R_s}=\frac{e_3-V_z}{R_z}
\end{align}
\begin{itemize}
	\item The current $i_c(t)$, flowing through the capacitor, is described by \eqref{eq:ideal_capacitor}
	\item The relationship between $i_c(t)$ and $i_s$ is prior established by KCL
	\item Further, the current $i_s$ flowing through the source resistor $R_s$ is equivalent to the current flowing through the Zener diode
\end{itemize}

The output voltage $e_3$ across the end terminals of the circuit is governed by the voltage division
\begin{equation}
    e_3=\frac{e_2R_z+V_zR_s}{R_s+R_z}
    \label{eq:noLoad_e3}
\end{equation}

As a consequence, the following expression describes the current flowing through resistor $R_s$, which can be calculated as:

\begin{equation}
    i_s=\frac{e_2-e_3}{R_s}=\frac{e_2-\bracket{\frac{e_2R_z+V_zR_s}{R_s+R_z}}}{R_s}=\frac{e_2(R_s+R_z)-e_2R_z-V_zR_s}{R_s(R_s+R_z)}=\frac{e_2-V_z}{(R_s+R_z)}
\end{equation}

The current through the capacitor can now be calculated using the following expression,
\begin{equation}
    i_c(t) = \sbracket{\frac{\frac{e_1 - e_2}{2} - V_{th}}{R_b}}\cdot u\bracket{\frac{e_1 - e_2}{2} - V_{th}}-\frac{e_2-V_z}{R_s+R_z}
\end{equation}

As a result, the differential equation that emerges from this is
\begin{equation}
    \frac{d(e_2)}{dt}=\frac{1}{C}\sbracket{\bracket{\frac{\frac{e_1 - e_2}{2} - V_{th}}{R_b}}\cdot u\bracket{\frac{e_1 - e_2}{2} - V_{th}}-\frac{e_2-V_z}{R_s+R_z}} \hspace{0.1cm} \text{for} \hspace{0.1cm} e_3 \geq V_z
    \label{eq:noLoad_diffEq2}
\end{equation}

