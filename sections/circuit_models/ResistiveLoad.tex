\subsubsection{Resistive Load}
 The analysis is performed under the condition of a resistive load, where only the primary switch in the multi-mode load is activated, thereby having only the resistor $R_{L_1}$ in the circuit. The equivalent circuit under this scenario is depicted in the figure

\begin{figure}[H]
	\centering
	
	\begin{circuitikz} [american voltages] \draw
    
    (0,0) to[vsourcesin, l=$V_{2}$] node[label=\textcircled{1}] {} (0,5)
    (0,5) to[battery1, l=$2V_{th}$] (1.9,5)
    (1.5,5) to[R,l=$2R_b$,i=$i_d$, v=$V_d$] (4.5,5)
    (4.5,5) to[nos,o-o] (6,5)
    
    (6,5) node[label=\textcircled{2}] {} to[capacitor, i=$i_c$, l=$C$,v=$V_c$] (6,0)
    
    (6,5) to[R,l=$R_s$,i=$i_s$] (9,5)
    
    (9,5) node[label=\textcircled{3}] {} to[battery1, l=$V_z$] (9,4)
    (9,4) to[R, l=$R_z$, i=$i_z$] (9,2)
    (9,2) to[nos,o-o] (9,0.5)
    (9,0.5) to[short] (9,0)
    
    (12,5) node[label=\textcircled{4}] {} to[R,l=$R_{L_1}$,v=$e_4$] (12,0)
    
    (9,5) to[short] (12,5)
    (0,0) to[short] (12,0)
    (6,0) node[ground]{} (5,-1);
    
    \end{circuitikz}
	
	\label{circ:resistive_load}
	\caption{Power supply circuit with a resistive load $R_{L_1}$}
\end{figure}

\paragraph{Nodal Analysis}
\subparagraph{Node \textcircled{2}}
Applying KCL to determine an expression for $i_d(t)$ at node \textcircled{2}
\begin{equation}
    \begin{split}
    	i_d(t) = i_c(t) + i_s
    \end{split}
\end{equation}
Apply the primary equation for $i_d(t)$ [Eq \ref{eq:diode_current}], the ideal capacitor model [Eq. \ref{eq:ideal_capacitor}] and Ohm's law,
\begin{equation}
	\sbracket{\frac{\frac{e_1 - e_2}{2} - V_{th}}{R_b}}\cdot u\bracket{\frac{e_1 - e_2}{2} - V_{th}} = C \frac{d(e_2)}{dt} + \frac{e_2 - e_3}{R_s}
	\label{eq:mode2_node2}
\end{equation}

\subparagraph{Node \textcircled{3}}
Applying KCL to determine an expression for $i_s(t)$ at node \textcircled{3}
\begin{equation}
    i_s(t) = i_z(t) + i_{L_1}
\end{equation}
\begin{equation}
    \frac{e_2-e_3}{R_s}= \sbracket{\frac{e_3 - V_z}{R_z}}\cdot u\bracket{e_3 - V_z} +\frac{e_3}{R_L}
\end{equation}
Note: $e_4 = e_3$ as there is no voltage drop between nodes \textcircled{4} and \textcircled{3}.

\subparagraph{Node \textcircled{4}}
Applying KCL to determine an expression for $i_{L_1}$ at node \textcircled{4}
\begin{equation}
	i_{L_1} = \frac{e_3}{R_L}
\end{equation}

\paragraph{Differential Equations}
The current $i_s$ depends on the biasing of the zener diode. The biasing of the zener is determined by the relation between $e_3$ and $V_z$. Two set of differential equations are derived based on the operation of the Zener.
\subparagraph{Cut off}
When $e_3 < V_z$ the zener is in cut off. There is no additional current flow at node \textcircled{3} as the switch remains open. Thus, the current flowing into node \textcircled{3} $i_s$ is just given by $$i_s = i_{l_1}$$
\begin{equation}
	\frac{e_2-e_3}{R_s}=\frac{e_3}{R_{L_1}}
	\label{eq:resistiveLoad_node3}
\end{equation}
In this state, the system is governed by these set of equations
\begin{equation}
	\begin{split}
    	i_c(t) + i_s &= \sbracket{\frac{\frac{e_1 - e_2}{2} - V_{th}}{R_b}}\cdot u\bracket{\frac{e_1 - e_2}{2} - V_{th}}\\
    	i_c(t)&=C\frac{d(e_2)}{dt}\\
    	i_s&=\frac{e_2-e_3}{R_s}=\frac{e_3}{R_{L_1}}
	\end{split}
	    \label{eq:resistiveLoad_system}
\end{equation}
Solving \eqref{eq:resistiveLoad_node3} for $e_3$ gives an expression for the voltage at node \textcircled{3}
\begin{equation}
	\begin{split}
		R_{L_1}e_2 - R_{L_1}e_3 = R_s e_3 \\
		R_{L_1}e_2 = (R_{L_1} + R_s)e_3 \\
		e_3 = \frac{R_{L_1}e_2}{R_{L_1} + R_s}
	\end{split}
	\label{eq:resistiveLoad_outputVoltage}
\end{equation}
Substitute $e_3$ into the expression for $i_s$ from \eqref{eq:resistiveLoad_system}
\begin{equation}
	\begin{split}
		i_s &= \frac{e_2 - \frac{R_{L_1}e_2}{R_{L_1} + R_s}}{R_s} 
			= \frac{e_2 (1 - \frac{R_{L_1}}{R_{L_1} + R_s})}{R_s} \\
			&= \frac{e_2 (\frac{(R_{L_1} + R_s) - R_{L_1}}{R_{L_1} + R_s})}{R_s} 
			= \frac{e_2}{R_{L_1} + R_s}
	\end{split}
	\label{eq:resistiveLoad_is}
\end{equation}
Hence, the current across the capacitor $i_c(t)$ is:
\begin{equation}
	i_c(t) = \sbracket{\frac{\frac{e_1 - e_2}{2} - V_{th}}{R_b}}\cdot u\bracket{\frac{e_1 - e_2}{2} - V_{th}} - \frac{e_2}{R_{L_1} + R_s}
\end{equation}
and it's resulting differential equation is,
\begin{equation}
	\frac{de_2}{dt} = \frac{1}{C}\bracket{\sbracket{\frac{\frac{e_1 - e_2}{2} - V_{th}}{R_b}}\cdot u\bracket{\frac{e_1 - e_2}{2} - V_{th}} - \frac{e_2}{R_{L_1} + R_s}}
	\label{eq:resistiveLoad_DiffEq1}
\end{equation}
This condition only holds true when $e_3 < V_z$, or, when
\begin{equation}
	\frac{R_{L_1}e_2}{R_{L_1} + R_s} < V_z
\end{equation}

\subparagraph{Reverse breakdown}
When $e_3 \geq V_z$ the zener is in reverse breakdown. The current flow in node \textcircled{3} must be accounted for as the switch remains closed. Thus, the current flowing into node \textcircled{3} $i_s$ is just given by $$i_s = i_{l_1} + i_z$$
\begin{equation}
	\frac{e_2-e_3}{R_s}=\frac{e_3}{R_{L_1}} + \frac{e_3 - V_z}{R_z}
	\label{eq:resistiveLoad_forwardBias_node3}
\end{equation}
In this state, the system is governed by these set of equations
\begin{equation}
	\begin{split}
    	i_c(t) + i_s &= \sbracket{\frac{\frac{e_1 - e_2}{2} - V_{th}}{R_b}}\cdot u\bracket{\frac{e_1 - e_2}{2} - V_{th}}\\
    	i_c(t)&=C\frac{d(e_2)}{dt}\\
    	i_s&=\frac{e_2-e_3}{R_s}=\frac{e_3}{R_{L_1}} + \frac{e_3 - V_z}{R_z}
	\end{split}
	    \label{eq:resistiveLoad_forwardBias_system}
\end{equation}
Solving \eqref{eq:resistiveLoad_forwardBias_node3} for $e_3$ gives an expression for the voltage at node \textcircled{3}
\begin{equation}
\begin{split}
R_{L_1} R_z e_2 - R_{L_1} R_z e_3 &= R_s R_z e_3 + R_s R_{L_1} (e_3 - V_z) \\
R_{L_1} R_z e_2 - R_s R_{L_1} V_z &= (R_{L_1} R_z + R_s R_z + R_s R_{L_1}) e_3 \\
e_3 &= \frac{R_{L_1} R_z e_2 + R_s R_{L_1} V_z}{R_s R_{L_1} + R_s R_z + R_{L_1} R_z}
\end{split}
\end{equation}
Substitute $e_3$ into the expression for $i_s$ from \eqref{eq:resistiveLoad_forwardBias_system}
\begin{equation}
\begin{split}
i_s &= \frac{e_2 - \left(\frac{R_{L_1}R_z e_2 + R_{L_1}R_s V_z}{R_s R_{L_1} + R_s R_z + R_{L_1} R_z}\right)}{R_s} \\
&= \frac{e_2 (R_s R_{L_1} + R_s R_z + R_{L_1} R_z) - R_{L_1}R_z e_2 - R_{L_1}R_s V_z}{R_s (R_s R_{L_1} + R_s R_z + R_{L_1} R_z)} \\
&= \frac{e_2 R_s R_{L_1} + e_2 R_s R_z + e_2 R_{L_1} R_z - R_{L_1}R_z e_2 - R_{L_1}R_s V_z}{R_s^2 R_{L_1} + R_s^2 R_z + R_s R_{L_1} R_z} \\
&= \frac{e_2 R_{L_1} + e_2 R_z - R_{L_1} V_z}{R_{L_1} R_z + R_s R_z + R_s R_{L_1}}
\end{split}
\end{equation}
Hence, the current across the capacitor $i_c(t)$ is:
\begin{equation}
	i_c(t) = \sbracket{\frac{\frac{e_1 - e_2}{2} - V_{th}}{R_b}}\cdot u\bracket{\frac{e_1 - e_2}{2} - V_{th}} - \frac{e_2 R_{L_1} + e_2 R_z - R_{L_1} V_z}{R_{L_1} R_z + R_s R_z + R_s R_{L_1}}
\end{equation}
and it's resulting differential equation is,
\begin{equation}
	\frac{de_2}{dt} = \frac{1}{C}\bracket{\sbracket{\frac{\frac{e_1 - e_2}{2} - V_{th}}{R_b}}\cdot u\bracket{\frac{e_1 - e_2}{2} - V_{th}} - \frac{e_2 R_{L_1} + e_2 R_z - R_{L_1} V_z}{R_{L_1} R_z + R_s R_z + R_s R_{L_1}}}
	\label{eq:resistiveLoad_DiffEq2}
\end{equation}
This condition only holds true when $e_3 \geq V_z$, or, when
\begin{equation}
	\frac{R_{L_1} R_z e_2 + R_s R_{L_1} V_z}{R_s R_{L_1} + R_s R_z + R_{L_1} R_z} \geq V_z
\end{equation}


