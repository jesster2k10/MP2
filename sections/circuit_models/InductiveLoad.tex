\subsubsection{Inductive Load}
The primary switch is off, and the secondary switch is on. The equivalent circuit has a resistive and inductive load connected in series

\begin{figure}[H]
	\centering
	
	\begin{circuitikz} [american voltages] \draw
    
    (0,0) to[vsourcesin, l=$V_{2}$] node[label=\textcircled{1}] {} (0,5)
    (0,5) to[battery1, l=$2V_{th}$] (1.9,5)
    (1.5,5) to[R,l=$2R_b$,i=$i_d$, v=$V_d$] (4.5,5)
    (4.5,5) to[nos,o-o] (6,5)
    
    (6,5) node[label=\textcircled{2}] {} to[capacitor, i=$i_c$, l=$C$,v=$V_c$] (6,0)
    
    (6,5) to[R,l=$R_s$,i=$i_s$] (9,5)
    
    (9,5) node[label=\textcircled{3}] {} to[battery1, l=$V_z$] (9,4)
    (9,4) to[R, l=$R_z$, i=$i_z$] (9,2)
    (9,2) to[nos,o-o] (9,0.5)
    (9,0.5) to[short] (9,0)
    
    (12,5) node[label=\textcircled{4}] {} to[R,l=$R_{L_2}$,i=$i_l$] (12,3)
    (12,3) to[inductor,l=$L$] (12,0)
    
    (9,5) to[short] (12,5)
    (0,0) to[short] (12,0)
    (6,0) node[ground]{} (5,-1);
    
    \end{circuitikz}
	
	\label{circ:inductive_load}
	\caption{Power supply circuit with an inductive load $L$}
\end{figure}

\paragraph{Nodal Analysis}
\subparagraph{Node \textcircled{2}}
Determine an expression for $i_d(t)$ at node \textcircled{2} using KCL
\begin{equation}
	\begin{split}
		i_d(t) &= i_c(t) + i_s \\
		\sbracket{\frac{\frac{e_1 - e_2}{2} - V_{th}}{R_b}}\cdot u\bracket{\frac{e_1 - e_2}{2} - V_{th}} &= C \frac{d(e_2)}{dt} + \frac{e_2 - e_3}{R_s}
	\end{split}
	\label{eq:inductiveLoad_node2}
\end{equation}

\subparagraph{Node \textcircled{3}}
Determine an expression for $i_s$ at node \textcircled{3} using KCL
\begin{equation}
	\begin{split}
		i_s &= i_z + i_l \\
		\frac{e_2 - e_3}{R_s} &= \frac{e_3 - V_z}{R_z} + i_{l_2}
	\end{split}
	\label{eq:inductiveLoad_node3}
\end{equation}

\subparagraph{Node \textcircled{4}}
Determine an expression for $i_l$ at node \textcircled{4} using KCL
\begin{equation}
	i_l = \frac{e_3 - e_4}{R_{L_2}}
	\label{eq:inductiveLoad_node4}
\end{equation}
\paragraph{Differential Equations}
The current $i_s$ depends on the biasing of the zener diode. The biasing of the zener is determined by the relation between $e_3$ and $V_z$. Two set of differential equations are derived based on the operation of the Zener.

\subparagraph{Cut Off}
When $e_3 < V_z$ the zener is in cut Off. There is no additional current flow at node \textcircled{3} as the switch remains open. Thus, the current flowing into node \textcircled{3} $i_s$ is just given by $$i_s = i_l$$
\begin{equation}
	\frac{e_2 - e_3}{R_s} = i_l
	\label{eq:inductiveload_il}
\end{equation}

Recalling the ideal inductor model in \eqref{eq:ideal_inductor} the rate of change of $i_{l}$ related to output voltage $e_4$ by,
\begin{equation}
	\begin{split}
		e_4 = L\frac{di_l}{dt} = e_3
	\end{split}
	\label{eq:inductiveLoad_output}
\end{equation}

In this state, the system is characterised by these set of equations
\begin{align}
	i_c(t) + i_s &= \sbracket{\frac{\frac{e_1 - e_2}{2} - V_{th}}{R_b}}\cdot u\bracket{\frac{e_1 - e_2}{2} - V_{th}} \\
	i_s &= i_l = \frac{e_2 - e_3}{R_s} = \frac{e_3 - e_4}{R_{L_2}} \\
	i_c(t) &= C \frac{de_2}{d_t} \\
	e_4 &= L\frac{di_l}{dt}
	\label{eq:inductiveLoad_system}
\end{align}
The voltage $e_4$ at node \textcircled{4} is given by KVL,
\begin{equation}
	e_4 = e_3 = e_2 - (R_s + R_{L_2})i_l
\end{equation}
Hence, the current across the capacitor $i_c(t)$ is:
\begin{equation}
	i_c(t) = \sbracket{\frac{\frac{e_1 - e_2}{2} - V_{th}}{R_b}}\cdot u\bracket{\frac{e_1 - e_2}{2} - V_{th}} - i_l
\end{equation}
and it's resulting differential equation is,
\begin{equation}
	\frac{de_2}{dt} = \frac{1}{C}\bracket{\sbracket{\frac{\frac{e_1 - e_2}{2} - V_{th}}{R_b}}\cdot u\bracket{\frac{e_1 - e_2}{2} - V_{th}} - i_l}
	\label{eq:inductiveLoad_DiffEq1}
\end{equation}
This condition only holds true when $e_3 < V_z$.
\\

Another differential equation can be found relating the rate of change of $i_l$ to the inductance from the system of equations in \eqref{eq:inductiveLoad_system}.
\begin{equation}
	\begin{split}
		L\frac{di_l}{dt} &= e_2 - (R_s + R_{L_2})i_l \\
		 \frac{di_l}{dt} &= \frac{1}{L}\bracket{e_2 - (R_s + R_{L_2})i_l}
	\end{split}
	\label{eq:inductiveLoad_DiffEq2}
\end{equation}
This also only holds true when $e_3 < V_z$, or, when
\begin{equation}
	e_2 - R_si_l < V_z
\end{equation}

\subparagraph{Reverse Breakdown}
When $e_3 \geq V_z$ the zener is in reverse breakdown. The current flow in node \textcircled{3} must be accounted for as the switch remains closed. Thus, the current flowing into node \textcircled{3} $i_s$ is just given by $$i_s = i_l + i_z$$
\begin{equation}
	\frac{e_2 - e_3}{R_s} = \frac{e_3 - V_z}{R_z} + i_l
	\label{eq:inductiveLoad_forwardBias_node3}
\end{equation}
In this state, the system is characterised by these set of equations
\begin{equation}
	\begin{split}
		i_c(t) + i_s &= \sbracket{\frac{\frac{e_1 - e_2}{2} - V_{th}}{R_b}}\cdot u\bracket{\frac{e_1 - e_2}{2} - V_{th}} \\
		i_s &= \frac{e_2 - e_3}{R_s} = \frac{e_3 - V_z}{R_z} + i_l \\
		i_c(t) &= C \frac{de_2}{d_t} \\
		e_4 &= L\frac{di_l}{dt} = e_3 - R_{L_2}i_l
	\end{split}
	\label{eq:inductiveLoad_system_forwardBias}
\end{equation}
Solving \eqref{eq:inductiveLoad_forwardBias_node3} for $e_3$ gives an expression for the voltage at node \textcircled{3}
\begin{equation}
	\begin{split}
		R_z e_2 - R_z e_3 = R_s e_3 - R_s V_z + R_z R_s i_l \\
		(R_s + R_z) e_3 = R_z e_2 + R_s V_z - R_z R_s i_l \\
		e_3 = \frac{R_z e_2 + R_s V_z - R_z R_s i_l}{R_s + R_z}
	\end{split}
	\label{eq:inductiveLoad_forwardBias_e3}
\end{equation}
Substitute $e_3$ into the expression for $i_s$ from \eqref{eq:inductiveLoad_system_forwardBias},
\begin{equation}
	\begin{split}
		i_s &= \frac{e_2 - \left(\frac{R_z e_2 + R_s V_z - R_z R_s i_l}{R_s + R_z}\right)}{R_s} \\
		    &= \frac{R_s e_2 - R_z e_2 - R_s V_z + R_z R_s i_l}{R_s + R_z} \\
		    &= \frac{e_2 - V_z + R_z i_l}{R_s + R_z}
	\end{split}
\end{equation}
Hence, the current across the capacitor $i_c(t)$ is:
\begin{equation}
	i_c(t) = \sbracket{\frac{\frac{e_1 - e_2}{2} - V_{th}}{R_b}}\cdot u\bracket{\frac{e_1 - e_2}{2} - V_{th}} - \frac{e_2 - V_z + R_z i_l}{R_s + R_z}
\end{equation}
and it's resulting differential equation is,
\begin{equation}
	\frac{de_2}{dt} = \frac{1}{C}\bracket{\sbracket{\frac{\frac{e_1 - e_2}{2} - V_{th}}{R_b}}\cdot u\bracket{\frac{e_1 - e_2}{2} - V_{th}} - \frac{e_2 - V_z + R_z i_l}{R_s + R_z}}
	\label{eq:inductiveLoad_DiffEq3}
\end{equation}
This condition only holds true when $e_3 \geq V_z$. Another differential equation can be found relating the rate of change of $i_l$ to the inductance from the system of equations in \eqref{eq:inductiveLoad_system_forwardBias}.
\begin{equation}
	\begin{split}
		L\frac{di_l}{dt} &= \frac{R_z e_2 + R_s V_z - R_z R_s i_l}{R_s + R_z} - R_{L_2}i_l \\
		 \frac{di_l}{dt} &= \frac{1}{L} \sbracket{\frac{R_z e_2 + R_s V_z - R_z R_s i_l}{R_s + R_z} - R_{L_2}i_l}
	\end{split}
	\label{eq:inductiveLoad_DiffEq4}
\end{equation}
This also only holds true when $e_3 \geq V_z$, or, when
\begin{equation}
	\frac{R_z e_2 + R_s V_z - R_z R_s i_l}{R_s + R_z} \geq V_z
\end{equation}

Thus, four differential equations \eqref{eq:inductiveLoad_DiffEq1}, \eqref{eq:inductiveLoad_DiffEq2}, \eqref{eq:inductiveLoad_DiffEq3}, \eqref{eq:inductiveLoad_DiffEq4} have been derived characterising the operation of the circuit under inductive load conditions.


