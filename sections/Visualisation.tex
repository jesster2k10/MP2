\section{Visualisation}
A dynamic user interface was created using MATLAB's GUIDE package to visualize the outcomes of the numerical method. The interface allows users to examine voltage nodes and measure the current across any component. When a specific element is probed, its voltage or current is plotted in real-time, resembling the behavior in LTspice. This functionality was implemented using MATLAB.

\subsection{User Interface}

*Include circuit  here


\begin{itemize}
    \item The user interface displayed in the figure exhibits several notable features, incorporating the circuit provided in the assignment brief with some modifications. These modifications include:
\end{itemize}

\begin{itemize}
    \item Incorporation of an ammeter in series with the bridge rectifier, this allows for the user to measure the current flowing through the diode $i_d$
\end{itemize}

\begin{itemize}
    \item Addition of a switch in series with the load resistor, allowing the user to connect or disconnect the load resistor as desired.
\end{itemize}

\begin{itemize}
    \item Introduction of a series combination of a switch, resistor, and inductor in parallel with the load resistor. This arrangement allows for a simpler way to connect and disconnect the inductive load as required.
\end{itemize}

\begin{itemize}
    \item Establishment of a ground node which is used as a reference in order to measure node voltages
\end{itemize}

\subsubsection{Measured Voltages}

labelling the essential nodes is done to allow for probing of voltages displayed below

$V_A$: The voltage across the capacitor relative to the ground.

$V_B$: The voltage across the Zener diode relative to the ground.

$V_C$: The voltage across the inductor relative to the ground.

$V_D$: The voltage across the resistive load relative to the ground.

$V_E$: The voltage across both the resistor $R_2$ and the inductor $L$ relative to ground.

Several new elements were then added for the simulation of the circuit. 

These elements are states as: 
\begin{itemize}
    \item Resistor and Inductor
When the "Resistor and Inductor" is enabled, both the inductive load and resistive load are connected.

    \item Resistor
When the "resistor" is enabled, the inductive load is disconnected and resistive load is connected. 

    \item Inductor
When the "Inductor" is enabled, the resistive load is disconnected and the inductive load is connected. 

    \item None 
When "None" is enabled, neither the resistive load or the inductive load are connected.

    \item Switch
\end{itemize}
  When "Switch" is enabled, the switches are connected and disconnected in a way to satisfy the following conditions:

"the power supply is switched on at time t = 0 at which time the smoothing capacitor voltage and inductor current are zero, neither attached loading device is active at this time; after 2 sec the resistive load is switched on; after a further 2.2 sec the inductive load is switched in (i.e. connected) and it remains connected for the next 1.263 sec; subsequently the inductive device is switched out (i.e. disconnected) and remains disconnected."

The addition of drop down menus were made to allow the change of the resistor $R_s$ and capacitor $C$ values with  a 10\% tolerance for the resistor and a 20\% tolerance for the capacitor. Knowing this, the values for the resistor $R_s$ are ($270\Omega$, $300\Omega$, $330\Omega$) with the $10\%$ tolerance. The values for the capacitor $C$ with the $20\%$ tolerance are $(1760\mu F, 2200\mu F, 2640\mu F)$ 

Using an extra plot button which has been added, it allows for the combination of multiple different locations of the circuit to be plotted simultaneously. some extra additions are also displayed below: 

\begin{itemize}
    \item When the mouse hovers over a node, the cursor changes to a symbol representing voltage, indicating the ability to probe the voltage at that node

    \item When the mouse hovers over a circuit element, the cursor changes to a symbol representing electric current, indicating the ability to probe the current through that element.

    \item The plot has two y-axes, one for voltage and one for current, allowing both to be displayed simultaneously.

    \item Clicking on a node without any plotted voltages or currents opens a plot where the probed element's voltage or current can be displayed over a 2-second period.

    \item When the plot is open, probing a different node adds its voltage or current to the existing plot instead of opening a new plot.

    \item If the same node is probed again while the plot is open, its voltage or current is not redundantly plotted.

    \item Switching between different modes closes the plot.

    \item Clicking the switching scenario button visually shows in real time how the switches are opened and closed according to the requirements of the switching scenario
\end{itemize}

*Sample plot figure should be here

\subsection{Implementation}
Using the GUIDE package in MATLAB, implementation of the user interface can be done 

The opening function is executed when the GUIDE interface is initially accessed. Its purpose is to define important global variables,  which hold the coordinates of all nodes and elements available for probing. Furthermore, the opening function specifies the appropriate functions to be triggered during cursor movement  and cursor click. The opening function saves data related to the current cursor which points downwards, the current cursor which points to the right side, and the voltage cursor.displayed below, enabling retrieval when the cursor hovers over them.


*Code needs to be included here


\subsection{Switching Modes}
In order to animate the switches' connection and disconnection based on different modes like "resistor" or "inductor" a straightforward technique was utilized. Separate images were created, representing each mode and featuring the switches positioned correctly for that mode. When any of the mode push buttons are clicked, the corresponding background image is displayed, and properly giving names to them. 

*code and images of the circuit need to be included here 

\subsection{Detection of Hovering}

Whenever the mouse moves, the function mouseMove() is called. By continuously monitoring the cursor's position, a simple check is performed to determine if it lies within the coordinates of any elements of interest stored in areaCoords. This check allows the identification of the specific element being hovered over. Since the type of each element in areaCoords is known, the corresponding cursor can be displayed accordingly within the mouseMove() function

\subsection{Detection of Clicking}
Similar to the hovering method, the clicked element can be identified using the same principles. The process involves combining the detection of the clicked element with logic to determine if it has been clicked before. If the element has been previously clicked, it is not plotted again. However, if it hasn't been clicked before, it is marked as plotted, and if it is the first element to be plotted, the figure is displayed.

* code needs to be included here

You don't need to call the RungeKuttaCircuit() function again if you have already plotted one node's voltage or current. Instead, you can simply add the relevant node to the existing plot. This means that as long as the plot figure is open, you can access the new data for plotting without calling the RungeKuttaCircuit() function again. The RungeKuttaCircuit() function is called through the realData() function.

*Code needs to be included here

The RungeKuttaCircuit() function produces data with a very small time step, causing the plot to be slow. However, most of this data is only used for the Runge-Kutta method and is not essential for visualization. To improve efficiency, only every 10th sample is extracted from these results. This sampled data is then dynamically plotted using the dynamicPlot() function, which utilizes the animatedline command. The plot includes two y-axes for voltage and current.

*Code needs to be included here

The dynamicPlot() function is a straightforward function capable of plotting multiple lines simultaneously. The number of plots corresponds to the size of the colors vector provided, which determines the color of each line. It is essential to specify the y-axis against which each line should be plotted since there are two y-axes available.

\subsubsection{Switching Scenario}

By combining the Runge-Kutta code for the switching scenario with the animation of switches through image changes at specific times, the switches are animated accordingly. This animation occurs when the Runge-Kutta Circuit code transitions from one mode to another.