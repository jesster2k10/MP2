\documentclass[11pt,a4paper,oneside]{report}

% MARK - Packages
\usepackage{amsmath}
\usepackage{siunitx}
\usepackage[x11names,table,xcdraw]{xcolor}
\usepackage{natbib}
\usepackage{mathrsfs}
\usepackage{color} %red, green, blue, yellow, cyan, magenta, black, white
\usepackage[scr]{rsfso}
\usepackage{graphicx}
\usepackage{listings}
\usepackage{bigfoot}
\usepackage[numbered,framed]{matlab-prettifier}
\usepackage{titletoc,tocloft}
\usepackage{setspace}
\usepackage{natbib}
\usepackage[a4paper,left=2cm,right=2cm,top=2.5cm,bottom=2.5cm]{geometry}
\usepackage[siunitx]{circuitikz}
\usepackage{framed}
\usepackage{quoting}
\usepackage{xlop}
\usepackage{float}
\usepackage{titlesec}
\usepackage{csvsimple}
\usepackage{tikz}
\usepackage{caption}
\usepackage{subcaption}
\usepackage[numbered,framed]{matlab-prettifier}
\usepackage{physics} % for 'pdv' macro
\usepackage{multicol}
\usepackage{courier}
\usepackage{sourcecodepro}
\usepackage[normalem]{ulem}
\usepackage{pgfplots}
\usepackage[bookmarks]{hyperref}
\pgfplotsset{width=10cm,compat=1.9}
\usepackage{url}
\usepackage{import}
\useunder{\uline}{\ul}{}

% MARK - Metadata
\author{
  Jesse Onolememen\\
  20344056
  \and
  Omar Issa\\
  19204747
  \and
  David Jones\\
  20329631
}
\title{
  \textbf{EEEN30150 Modelling \& Simulation} \\
  \large Minor Project 2 (MP2): Dynamic Equations}
\date{\today}
\onehalfspacing
\setcounter{secnumdepth}{4}
\setcounter{tocdepth}{3}
\graphicspath{../exports}
\bibliographystyle{agsm}
\numberwithin{equation}{section}
\setlength{\parindent}{0pt}

% MARK - Quotations
\colorlet{shadecolor}{LavenderBlush2}
\colorlet{framecolor}{lightgray}
\renewenvironment{shaded*}{%
  \def\FrameCommand{\setlength{\FrameRule}{2pt}\fboxsep=\FrameSep \fcolorbox{framecolor}{shadecolor}}%
  \MakeFramed {\advance\hsize-\width \FrameRestore}}%
 {\endMakeFramed}


\newenvironment{shquote}
 {\begin{shaded*}
  \quoting[leftmargin=0pt, vskip=0pt]
 }
 {\endquoting
 \end{shaded*}
}

% Title format
\titleformat{\chapter}[display]
  {\normalfont\bfseries}{}{0pt}{\Huge}
\titlespacing*{\chapter}{0pt}{0pt}{12pt}

% MARK - Custom Commands
\setcounter{section}{0}
\renewcommand{\thesection}{\arabic{section}}
\newcommand{\lpt}[1]{\mathcal{L}\{#1\}}
\newcommand{\sbracket}[1]{\left[#1 \right]}
\newcommand{\bracket}[1]{\left(#1 \right)}

\def\therefore{\boldsymbol{\text{ }
\leavevmode
\lower0.4ex\hbox{$\cdot$}
\kern-.5em\raise0.7ex\hbox{$\cdot$}
\kern-0.55em\lower0.4ex\hbox{$\cdot$}
\thinspace\text{ }}}

% Listings
\lstset{style=Matlab-editor}
%\setmonofont{Lucida Console}[Scale=MatchLowercase] % select a suitable monospaced font
\lstset{basicstyle=\ttfamily\footnotesize,breaklines=true}
\lstset{framextopmargin=50pt,frame=bottomline}

\usepackage{xcolor}
\definecolor{maroon}{cmyk}{0, 0.87, 0.68, 0.32}
\definecolor{halfgray}{gray}{0.55}
\definecolor{ipython_frame}{RGB}{207, 207, 207}
\definecolor{ipython_bg}{RGB}{247, 247, 247}
\definecolor{ipython_red}{RGB}{186, 33, 33}
\definecolor{ipython_green}{RGB}{0, 128, 0}
\definecolor{ipython_cyan}{RGB}{64, 128, 128}
\definecolor{ipython_purple}{RGB}{170, 34, 255}


%%
%% Python definition (c) 1998 Michael Weber
%% Additional definitions (2013) Alexis Dimitriadis
%% modified by me (should not have empty lines)
%%
\lstdefinelanguage{MatLab}{
    basicstyle=\footnotesize,
    keywordstyle=\color{ipython_green}\ttfamily,
}

% Matrix columns
\setcounter{MaxMatrixCols}{15}
\newcommand{\nodecirc}[1]{\textcircled{#1} \hspace{0.15cm} \triangleright \hspace{0.25cm}}

% MARK - Begin Document

\begin{document}

% S; Front Page
\maketitle

% S; Plagarism Declaration
\leavevmode%
\vfill\noindent
\begin{center}
  \textbf{Declaration of Authorship}
\end{center}
I declare that all material in this assessment is my our work except where there is clear acknowledgement and appropriate reference to the work of others.

\vspace*{4em}\noindent
\hfill%
\begin{tabular}[t]{c}
  \rule{10em}{0.4pt}\\ Signed
\end{tabular}%
\hfill%
\begin{tabular}[t]{c}
\rule{10em}{0.4pt}\\ Date
\end{tabular}%
\hfill
\strut
\vspace{4em}


% S; Table of Contents
\newpage
\setlength{\cftsubsecindent}{1.5cm}
\setlength{\cftsubsubsecindent}{1.5cm}
\doublespacing
\tableofcontents
\onehalfspacing

% S; Sections
\section*{Abstract}
\subsection*{Problem 2.1: A Simple Power Supply}
A simple power supply is presented for analysis. 
\begin{figure}[H]
    \centering
    \includegraphics[width=\textwidth]{graphics/powersupply.png}
    \caption{Simple power supply}
    \label{fig:powersupply}
\end{figure}
The power supply is terminated by a multi-mode load resistor modelled by its Thevenin equivalent circuit. The circuit has multiple modes of operation depending on the switching of its devices. The circuit is presented below.
\begin{figure}[H]
    \centering
    \includegraphics[width=8cm]{graphics/multimode_load.png}
    \caption{Multi-mode load}
    \label{fig:multimode_load}
\end{figure}
The performance of the circuit is to be determined for all modes of operation. A mathematical model of the circuit is derived for use in numerical analysis and simulation. A visualisation is presented at the end with the results.
\section{Modelling}
A model of each individual circuit component is fabricated. This is used to provide a model of the circuit as a whole, using nodal analysis methods .i.e. Krichoff's Laws

\subsection{Component Values}
The values of the circuit components are presented in the table below. Where there is an applicable tolerance level, the maximum and minimum values are shown. 
\begin{table}[H]
    \centering
    \begin{tabular}{|c|c|c||c|c|}\hline
        & Value & \% Tolerance & \textbf{Maximum} & \textbf{Minimum} \\\hline
       $C$ & $0.0022 F$ & $20\%$ & $0.00264F$ & $0.00176F$ \\
       $R_s$ & $300\Omega$ & $10\%$ & $330\Omega$ & $270\Omega$ \\
       $R_{L_1}$ & $65.7\Omega$ & $0\%$ & $65.7\Omega$  & $65.7\Omega$ \\
       $R_{L_2}$ & $287\Omega$ & $0\%$ & $287\Omega$ & $287\Omega$ \\
       $R_b$ & $3.2 \Omega$ & $0\%$ & $3.2 \Omega$ & $3.2 \Omega$ \\
       $L_L$ & $0.01058 H$ & $0\%$ & $0.01058 H$ & $0.01058 H$ \\\hline
    \end{tabular}
    \caption{Circuit component values and their variances}
    \label{tab:component_values}
\end{table}

\subsection{Component Modelling}
Each individual component in the circuit must be represented analytically. Circuit components are assumed to be ideal, .i.e. there is an absence of source resistance and or impedance. The model does not account for inexactitudes in component values that may be realised in a practical implementation. 

\subsubsection{Voltage Source} The voltage source is modelled as an ideal AC voltage source. There is no source impedance, thus, we assume that the source generates the exact amount of voltage across its terminals. 
\begin{equation}
    \begin{split}
        V_{\text{in}} &= V_{\text{rms}}\sqrt{2}\text{sin}(\omega t + \phi)
            \\ &= 230\sqrt{2}\text{sin}(100\pi t)V
    \end{split}
\end{equation}
where $V_{\text{rms}}$ is the rms voltage, $\omega = 2\pi f$ is the angular frequency

\subsubsection{Capacitor} The capacitor is modelled as an ideal capacitor that does not dissipate energy. The current across the capacitor is given by:
\begin{equation}
    i = C \frac{dV}{dt}
    \label{eq:ideal_capacitor}
\end{equation}
where $C$ is the capacitance in Farads (F), $\frac{dV}{dt}$ is the rate of change of voltage $\left(\frac{V}{s}\right)$

\subsubsection{Inductor} The inductor is modelled as an ideal inductor whose voltage is given by:
\begin{equation}
    V = L\frac{di}{dt}
\end{equation}
where $L$ is the inductance in Henry's, $\frac{di}{dt}$ is the rate of current change
\subsubsection{Resistor} The resistor is modelled as an ideal resistor whose voltage is given by Ohm's law
\begin{equation}
    V=IR
\end{equation}
where $V$ is the voltage (V), $I$ is the current (A) and $R$ is the resistance ($\Omega$)

\subsubsection{Switch} The switch is modelled as an ideal switch without any source resistance. The switch can be represented using a binary state .i.e. on or off. It is assumed that there is no propagation delay in the change of state and the effects are immediate. 

\subsubsection{Transformer} The transformer is assumed to be ideal with a turns ratio of $20:1$. The voltage at either side of the transformer can be determined from its common ideal representation
\begin{equation}
    \frac{V_1}{V_2} = \frac{N_1}{N_2}
\end{equation}
where $N_1=20$ is the number primary turns, $N_2=1$ is the number of secondary turns, $V_1$ is the primary voltage, $V_2$ is the secondary voltage.\\

The voltage at LV-side of the transformer is fed into the full-bridge rectifier. The secondary voltage $V_2$ is given by:
\begin{equation}
    V_2 = \frac{V_1N_2}{N_1}
\end{equation}

\subsubsection{Diode}
The four diodes in the circuit are configured to form a full bridge rectifier. It is assumed all four diodes are identical.

\paragraph{Shockley Ideal Diode}
The diodes can be modelled as a Shockley diode in series with bulk resistance. The Shockley diode model is a simplified model to describe the behaviour of a diode:

\begin{equation}
	I_D = I_s\left( e^{\frac{V_D}{\eta V_T}} - 1 \right) + \frac{V_d}{R_b}	\label{shockeyDiodeWithBulkResistance}
\end{equation}
where,
\begin{itemize}
	\item $I_D$ is the diode current
	\item $I_s$ is the saturation current
	\item $V_D$ is the voltage across the diode
	\item $\eta$ is the ideality factor
	\item $V_T$ is the thermal voltage (determined by $V_T = \frac{KT}{q}$)
	\item $R_b$ is the bulk resistance
\end{itemize}	

The Shockley diode presents a non-linear model of the diode's behaviour which complicates circuit analysis. A piece-wise linear approximation to the DC characteristic of the diode is used for simplicity.

\paragraph{Piece-wise Linear Approximation}
An ideal diode behaves as a switch–conducting current without any losses under forward bias, and completely blocking current under reverse bias. The biasing of a diode is determined by assessing whether the voltage drop across the diode is greater than its threshold voltage ($V_D \geq V_{th}$)

\subparagraph{Ideal Diode}
Assume that the diode is a silicon diode with a threshold voltage of $V_{th} = 0.7 V$.  The current across diode can be represented analytically as:
\begin{equation}
    i_d = \begin{cases}
        \infty \hspace{1cm} &V_d \geq 0.7V \\
        0 \hspace{1cm}   &V_d < 0.7V
    \end{cases}
    \label{eq:ideal_diode}
\end{equation}
\begin{figure}[H]
    \centering
    \includegraphics[width=10cm]{graphics/diode_iv.png}
    \caption{The ideal diode i-v characteristic. This graph assumes $V_{th}=0$}
    \label{fig:diode_iv_characterstic}
\end{figure}
The diode is represented as a cell ($V_{th}$) in series with a resistor ($R_b$) and an ideal switch. The current flowing into the switch is represented by $i_d$. This is the piece-wise linear approximation of the diode.
\begin{figure}[H]
    \centering
    \begin{circuitikz}[american voltages] \draw
 	(0,5) to[diode,o-o,l=$V_d$,i=$i_d$] (0,0)
	(3,5) to[battery1,o-,l=$V_{th}$] (3,4) 
	(3,4) to[R, l=$R_b$, i_=$i_d$] (3,1)
	(3,1) to[nos,o-o] (3,0);
\end{circuitikz}
    \caption{Piece-wise linear diode approximation of the ideal diode}
    \label{fig:piecewise_diode}
\end{figure}

Analytically $i_d$ can be described using Ohm's law and considering the voltage drop across $R_b$ 
\begin{equation}
    i_d = \begin{cases}
        \bracket{\frac{V_d - V_{th}}{R_b}} \hspace{1cm} &V_d \geq 0.7V \\
        0 \hspace{1cm}   &V_d < 0.7V
    \end{cases}
\end{equation}
This expression matches the expression for the ideal diode presented in Equation \ref{eq:ideal_diode}. 

\subparagraph{Determination of Maximum Current}
To accurately model the diode using the piece-wise linear circuit in Figure \ref{fig:piecewise_diode}, a reasonable estimate of the maximum current flowing through the diode while it is forward biased must be made.\\

Consider the behaviour of the diode inside the bridge rectifier:

\begin{figure}[H]
	\centering
	
    \begin{circuitikz} [american voltages] \draw
    
    (0,0) to[vsourcesin, l=$V_{in}$] (0,4)
    (0,4) to[battery1, l=$2V_{th}$] (2,4)
    (2,4) to[R,l=$2R_b$,i=$i_d$, v=$V_d$] (7,4)
    (7,4) to[nos,o-o] (8,4)
    (8,4) to[short] (10,4)
    (10,4) to[capacitor, l=$C$,v=$V_c$] (10,0)
    (0,0) to[short] (10,0)
    (5,0) node[ground]{} (5,-1);
    
    \end{circuitikz}
	
	\caption{Piece-wise linear diode model applied in the half-rectifier}
	\label{fig:diode_circuit}
\end{figure}

\begin{itemize}
	\item In a full-bridge rectifier, two diodes are used. One for each half of the cycle (.i.e. half the period of the input sinusoid). Thus, this inequality must be satisfied to ensure forward biasing $$V_d > 2V_{th} = V_d > 1.4V$$
	\item The rectifier is being connected to a capacitor in Figure \ref{fig:powersupply}. When a capacitor is uncharged (.i.e. $V=0$) it behaves as a short circuit. In a DC circuit, the capacitor experiences maximal current at the instant the power supply is connected.
	\item In an AC circuit (.i.e. with a sinusoidal input), the voltage changes over time, so the capacitor does not experience a sudden surge of voltage as it does in the DC circuit.
	\item The maximum charging current occurs during the first peak of the sinusoidal wave, which occurs in the first half cycle: $$0 < t < \frac{T}{2}$$ where T is the period of the sinusoid $T=\frac{1}{f}$
\end{itemize} 

The rate of change of the capacitor voltage ($\frac{dV_c}{dt}$) is the difference across the input and output terminals,
\begin{align}
	\frac{dV_c}{dt} = \frac{V_{in}(t) - 2V_{th} - V_c}{2R_b}
	\label{eq:dvc_dt}
\end{align}
The input voltage $V_{in}(t)$ is stepped down by a factor of 20 due to the rectifier being connected to the LV-side of the ideal transformer (see Figure \ref{fig:powersupply}). Hence it's amplitude $A$ is given by $A = \frac{230\sqrt{2}}{20}$.\\

Substituting [\ref{eq:dvc_dt}] into the ideal capacitor model in [\ref{eq:ideal_capacitor}], derive a first-order ODE of $V_c(t)$
\begin{align}
	2R_bC\frac{dV_c}{dt} + V_c(t) = A\text{sin}(\omega t) - 1.4
	\label{eq:diff}
\end{align}
where $A$ is the amplitude of $V_{in}(t)$, $\omega = 100\pi$. To solve the ODE, split it into homogenous and non-homogenous parts. \\

The homogenous part of this equation (.i.e. excluding the sinusoidal term) is:
\begin{equation}
	2R_bC\frac{dV_c}{dt} + V_c(t) = 0
\end{equation}
which has a solution in the form
\begin{equation}
	V_c(t) = Be^{-t\bracket{\frac{1}{2R_bC}}}
\end{equation}
where $B$ is an arbitrary constant to be determined by the initial conditions. 
\\

The non-homogenous part of the equation is defined over the interval $0 \leq t \leq \frac{T}{2}$ and is a sinusoid:
\begin{equation}
	Q\frac{dV_c}{dt} + V_c(t) = A\text{sin}(\omega t) - 1.4
\end{equation}
To find the particular solution, assume a solution in the form:
\begin{equation}
	V_c(t) = M\text{sin}(\omega t) + N\text{cos}(\omega t) 
	\label{vc_t}
\end{equation}
where $M$ and $N$ are arbitrary constants to be determined by initial conditions.\\

The solution to the ODE $V_c(t)$ is the combination of the homogenous and non-homogenous solutions
\begin{equation}
	V_c(t) = Be^{-t\bracket{\frac{1}{2R_bC}}} + M\text{sin}(\omega t) + N\text{cos}(\omega t) - 1.4
\end{equation}
At time $t=0$ the capacitor is uncharged. Thus, the initial condition holds $V_c(0) = 0$

Define $p$ as $2R_bC$. The given ordinary differential equation [\ref{eq:diff}] can then be written as:
\begin{align}
p\frac{dV_c}{dt} + V_c(t) = A\sin(\omega t) - 1.4
\end{align}

The initial condition $V_c(0) = 0$ translates to:
\begin{equation}
B + N - 1.4 = 0
\end{equation}

The derivative of this non-homogeneous part of the ODE [\ref{vc_t}] is:
\begin{equation}
\frac{dV_c}{dt} = M\omega\cos(\omega t) - N\omega\sin(\omega t)
\end{equation}

Substituting the derivative of $V_c(t)$ and $V_c(t)$ into the non-homogeneous ODE yields:
\begin{align}
p(M\omega\cos(\omega t) - N\omega\sin(\omega t)) + (M\sin(\omega t) + N\cos(\omega t)) &= A\sin(\omega t) - 1.4
\end{align}

This simplifies to:
\begin{align}
pM\omega\cos(\omega t) - pN\omega\sin(\omega t) + M\sin(\omega t) + N\cos(\omega t) &= A\sin(\omega t) - 1.4
\end{align}

To satisfy this equation for all $t$, the coefficients of $\sin(\omega t)$ and $\cos(\omega t)$ on the left-hand side must match those on the right-hand side. This results in a system of equations:
\begin{align}
pM\omega + N &= 0\\
-pN\omega + M &= A
\end{align}

Solving this system yields $M$ and $N$. Substituting $M$ into the first equation provides $N$:
\begin{align}
N &= -pM\omega\\
-p(-pM\omega)\omega + M &= A\\
p^2M\omega^2 + M &= A
\end{align}

Solving for $M$ gives:
\begin{equation}
M = \frac{A}{p^2\omega^2 + 1}
\end{equation}

Substituting $M$ back into the first equation yields $N$:
\begin{equation}
N = -pM\omega = -p\frac{A}{p^2\omega^2 + 1}\omega = -\frac{pA\omega}{p^2\omega^2 + 1}
\end{equation}

Using the initial condition $V_c(0) = 0$ and the equation $B + N - 1.4 = 0$, the value of $B$ is found to be:
\begin{equation}
B = 1.4 - N = 1.4 + \frac{pA\omega}{p^2\omega^2 + 1}
\end{equation}

Consequently, the constants $B$, $M$, and $N$ are found to be:
\begin{align}
B &= 1.4 + \frac{pA\omega}{p^2\omega^2 + 1}\\
M &= \frac{A}{p^2\omega^2 + 1}\\
N &= -\frac{pA\omega}{p^2\omega^2 + 1}
\end{align}
Substitute the values for $p=2R_bC=0.01408$, $A=\frac{230\sqrt{2}}{20}$ and $\omega=100\pi$:
\begin{align}
	B &= 1.4 + \frac{(0.01408)(\frac{230\sqrt{2}}{20})(100\pi)}{(0.01408)^2(100\pi)^2 + 1}\\
	M &= \frac{\frac{230\sqrt{2}}{20}}{(0.01408)^2(100\pi)^2 + 1}\\
	N &= -\frac{(0.01408)(\frac{230\sqrt{2}}{20})(100\pi)}{(0.01408)^2(100\pi)^2 + 1}
\end{align}
Thus, the constants $B, M, N$ are expressed numerically as,
\begin{align}
	B &= 4.897 \\
	M &= 0.7907 \\
	N &= -3.497
\end{align}
Finally derive the solution for $V_c(t)$:
\begin{equation}
	V_c(t) = 4.897e^{-71.0227t} + 0.7907\text{sin}(100\pi t) -3.497\text{cos}(100\pi t) - 1.4
	\label{eq:capacitor_voltage}
\end{equation}
Hence, determine the capacitor current $i_c(t)$ referring to Equation \ref{eq:ideal_capacitor}
\begin{equation}
	\begin{split}
		i_c(t) &= 0.0022\sbracket{-71.0227(4.897)e^{-71.0227t} + 100\pi*0.7907\cos(100\pi t) + 100\pi*3.497\sin(100\pi t)} \\
			   &=-0.7651e^{-71.0227t} + 0.5464\cos(100\pi t) + 2.41695\sin(100\pi t)
	\end{split}
\end{equation}

\pagebreak
The maximum current can be determined graphically by plotting $i_c(t)$ over the interval $0 < t < \frac{T}{2}$
\begin{figure}[H]
	\centering
	\includegraphics[width=\textwidth]{graphics/diode_current.png}
	\caption{$V_c(t)$ and $i_c(t)$ plot over a half period}
	\label{fig:diode_current}
\end{figure}
The graph above gives a reasonable approximation of the maximum current $1.917A$ flowing through the signal diode in forward bias. The piecewise-linear diode model approximates the behaviour of the Shockley diode for a current range up to $\simeq 3.834A$.

\subsubsection{Zener Diode}


\subsection{Circuit Modelling}
Based on the previously discussed assumptions and Figure (1.9), the circuit can be reconfigured as depicted in Figure (2). To elucidate the performance of the circuit in all four load modes, the following equations can be employed:\\

[insert circuit diagram 1.7]\\

\emph{Figure 1.7: Power supply circuit with multimode load $R_L$}

[insert circuit diagram 1.8]\\

\emph{Figure 1.8: Modified circuit with implemented model of circuit elements}\\

The current analysis of the diodes involves utilizing the unit step function, which acts as a switch in the model. When in the on-state, the unit step function is set to 1, and when in the off-state, it is set to 0. This function can be defined as follows:

\begin{equation}
    i_d=u(\frac{V_2-V_a}{2}-V_t)(\frac{\frac{V_2-Va}{2}-V_t}{R_d})
\end{equation}\\

Likewise, the current flowing through the Zener diode:\\

\begin{equation}
    i_z=(\frac{V_b-V_z}{R_z})[u(V_b-V_z)]
\end{equation}\\

\large\textcolor{red}{Mode 1: The load does not include an inductor or resistor (both components are deactivated)}:\\

[insert figure 1.9]\\

\emph{Figure 1.9: Circuit with no load (both devices off)}\\

\textcolor{blue}{KCL at node A:}

\begin{equation}
    i_d=i_c+i_s
    u(\frac{V_2-V_a}{2}-V_t)(\frac{\frac{V_2-Va}{2}-V_t}{R_d})=C(\frac{dV_a}{dt}+\frac{V_a-V_b}{R_s}
\end{equation}\\

where u represents the unit step response.

\textcolor{blue}{KCL at Node B:}
\begin{equation}
    i_s=i_z
    (\frac{V_a-V_b}{R_s}=\frac{V_b-V_z}{R_z}u(V_b-V_z)
\end{equation}\\

Therefore, it is evident that when no load is connected, the current through the Zener can be determined by:

\begin{equation}
    i_z= \begin{cases}
    \frac{V_a-V_b}{R_s}=0,  V_b<V_z\\
    \frac{V_a-V_b}{R_s}=\frac{V_b-V_z}{R_z},  V_b\geq V_z
    \end{cases}
\end{equation}\\

When $V_b$ < $V_z$, $V_a$ is equal to $V_b$, and in this scenario, the current through the Zener can be expressed as:

\begin{equation}
    i_z=\begin{cases}
        0,  V_a<V_z \\
        \frac{V_a-V_b}{R_s}=\frac{V_b-V_z}{R_z},  V_a>V_z
    \end{cases}
\end{equation}

Therefore, if $V_a < V_z$, the Zener diode is in the cut-off region. In this case, the system can be described by the following expressions:\\

\begin{equation}
    \begin{cases}
        i_c=C\frac{dV_a}{dt}\\
        u(\frac{V_2-V_a}{2}-V_{th})(\frac{\frac{V_2-V_a}{2}-V_{th}}{R_d})=i_c
    \end{cases}
\end{equation}\\

Consequently, the differential equation that arises from this is:

\begin{equation}
    \frac{dV_a}{dt}=\frac{1}{C}u(\frac{V_2-V_a}{2}-V_{th})(\frac{\frac{V_2-V_a}{2}-V_{th}}{R_d}),  \emph{if $V_a<V_z$}
\end{equation}

When $V_a \geq V_z$, the Zener diode is in reverse bias. In this situation, the system can be characterized by the following expressions:

\begin{equation}
    \begin{cases}
        i_c=C\frac{dV_a}{dt}\\
        u(\frac{V_2-V_a}{2}-V_{th})(\frac{\frac{V_2-V_a}{2}-V_{th}}{R_d})=i_c+i_s\\
        i_s=\frac{V_a-V_b}{R_s}=\frac{V_b-V_z}{R_z}
    \end{cases}
\end{equation}\\

Thus, it is possible to determine the output voltage, denoted as $V_b$:\\

\begin{equation}
    V_b=\frac{V_aR_z+V_zR_s}{R_s+R_z}
\end{equation}

As a consequence, the following expression describes the current flowing through resistor $R_s$, which can be calculated as:

\begin{equation}
    i_s=\frac{V_a-V_b}{R_s}=\frac{V_a-\frac{V_aR_z+V_zR_s}{R_s+R_z}}{R_s}=\frac{V_a(R_s+R_z)-V_aR_z-V_zR_s}{R_s(R_s+R_z)}=\frac{V_a-V_z}{(R_s+R_z)}
\end{equation}

The current through the capacitor can now be calculated using the following expression:\\

\begin{equation}
    i_c=u(\frac{V_2-V_a}{2}-V_t)(\frac{\frac{V_2-V_a}{2}-V_{th}}{R_d})-\frac{V_a-V_z}{(R_s+R_z)}
\end{equation}


As a result, the differential equation that emerges from this is:

\begin{equation}
    \frac{dV_a}{dt}=\frac{1}{C}\left(u\left(\frac{V_2-V_a}{2}-V_t\right)\left(\frac{\frac{V_2-V_a}{2}-V_{th}}{R_d}\right)-\frac{V_a-V_z}{R_s+R_z}\right), \text{if } V_a \geq V_z
\end{equation}\\

\large\textcolor{red}{\emph{Mode 2: When a resistor is present at the load, the circuit depicted in Figure (1.9) can be simplified as follows:}}\\

[insert figure 2]\\

\emph{Figure 2 illustrates the revised circuit configuration when a resistor load is connected at the output, with one device activated.}\\

\textcolor{blue}{KCL at Node A:}\\

\begin{equation}
    u(\frac{V_2-V_a}{2}-V_{th})(\frac{\frac{V_2-V_a}{2}-V_{th}}{R_d}=C(\frac{dV_a}{dt}+\frac{V_a-V_b}{R_s}
\end{equation}\\

\textcolor{blue}{KCL at Node B:}\\

\begin{equation}
    \frac{V_a-V_b}{R_s}=\frac{V_b-V_z}{R_z}u(V_b-V_z)+\frac{V_b}{R_L}
\end{equation}\\

Therefore, it is evident that when a resistor is connected, the current flowing through resistor Rs can be expressed as:
\begin{equation}
    I_{R_s} = \frac{V_{in} - V_a}{R_s}
\end{equation}


where $V_{in}$ is the input voltage and $V_a$ is the voltage across the load resistor.\\

\begin{equation}
    i_s=\begin{cases}
        \frac{V_a-V_b}{R_s}=\frac{V_b}{R_L},  V_b<V_z\\
        \frac{V_a-V_b}{R_s}=\frac{V_b-V_z}{R_z}+\frac{V_b}{R_L},  V_b \geq V_z
    \end{cases}
\end{equation}\\

When $V_b$ < $V_z$, the voltage $V_b$ can be expressed as $\frac{R_LV_a}{R_s+R_L}$. Therefore, the above expression can be equivalently written as:

\begin{equation}
    \begin{cases}
        \frac{V_a-V_b}{R_s}=\frac{V_b}{R_L}, \frac{R_LV_a}{R_s+R_L}<V_z\\
        \frac{V_a-V_b}{R_s}=\frac{V_b-V_z}{R_z}+\frac{V_b}{R_L}, \frac{R_LV_a}{R_s+R_L} \geq V_z
    \end{cases}
\end{equation}\\

Hence, if $\frac{R_LV_a}{R_s+R_L}<V_z$, the Zener diode is in cut-off mode. In this case, the system can be described by the following expressions:

\begin{equation}
    \begin{cases}
        
    \end{cases}
\end{equation}



\section{Numerical Analysis}
\subsection{Classic Runge-Kutta Method}

Belonging to a family of iterative methodologies, Runge-Kutta methods, particularly the fourth-order variant (RK4), serve as prominent tools for numerically addressing differential equations. RK4 finds widespread use due to its efficacy in generating solutions to ordinary differential equations (ODEs) that may otherwise be challenging or impractical to solve analytically.

The crux of the Runge-Kutta method lies in using derivative evaluations at several junctures within a designated time step to project an estimate for the function's value at the succeeding step. This is achieved by approximating the Taylor series up to the order of the step size, which is conventionally represented by 'h'.

To illustrate, for a first-order ODE defined as $y' = f(t, y)$, the application of RK4 to progress from time $t$ to $t + h$ would involve the following steps:

\begin{equation}
\begin{aligned}
k_1 &= h \cdot f(t, y), \\
k_2 &= h \cdot f(t + \frac{h}{2}, y + \frac{k_1}{2}), \\
k_3 &= h \cdot f(t + \frac{h}{2}, y + \frac{k_2}{2}), \\
k_4 &= h \cdot f(t + h, y + k_3), \\
y(t + h) &= y(t) + \frac{1}{6} \cdot (k_1 + 2k_2 + 2k_3 + k_4).
\end{aligned}
\end{equation}

In this instance, the calculated 'k' values represent estimations of the function's slope at diverse points within the time step. The method essentially involves crafting a weighted mean of these slopes to derive the next value.\\

Let us define the next step in the solution, $y_{n+1}$, by the equation
\begin{equation}
y_{n+1} = y_n + \frac{h}{6} (k_1 + 2k_2 + 2k_3 + k_4) \quad (3.1)
\end{equation}
Here, $h$ represents the increment in the value of $t$ at each step.

The coefficients $k_i$ denote the slopes at different points in the time step and are defined as follows:
\begin{equation}
f(t, y) = \frac{dy}{dt}\bigg|_{(t, y)} = y'(t, y) \quad (3.2)
\end{equation}
\begin{align*}
k_1 &= f(t_n, y_n), &\text{initial slope using Euler's method},\\
k_2 &= f \left(t_n + \frac{h}{2}, y_n + \frac{k_1}{2}\right), &\text{slope at the midpoint using $y$ and $k_1$},\\
k_3 &= f \left(t_n + \frac{h}{2}, y_n + \frac{k_2}{2}\right), &\text{slope at the midpoint using $y$ and $k_2$},\\
k_4 &= f(t_n + h, y_n + k_3), &\text{slope at the end of the interval using $y$ and $k_3$}.
\end{align*}

These coefficients are defined for $n = 0, 1, 2, 3, \ldots$ . It can be clearly observed that this method bestows higher weights to the midpoint slopes during the computation of their weighted average.

RK4 can be conveniently extended to cater to systems of ordinary differential equations by replacing $y_n$ with the corresponding vector of unknowns.

Four distinctive scenarios are to be considered, each featuring diverse loads connected to the DC rectifier. These will be thoroughly analyzed individually. In the derivation of the equations, an emphasis is placed on optimizing them to incorporate a minimal number of operations. The priority order is addition, multiplication, and division, enabling a reduction in the utilized machine cycles.

\begin{center}
\Large Establishing Initial Conditions for the Runge-Kutta Algorithm
\end{center}\\

The Runge-Kutta algorithm, like any other method used for numerically solving ordinary differential equations, requires predefined initial conditions to initiate its iteration process. These initial conditions serve as the springboard for the subsequent calculations. In the context of an electrical circuit, these conditions are established based on the state of the system at the start time, typically denoted as t = 0s.

For a circuit that has been left in a stationary state before power supply is connected, we can infer that all transient effects have subsided and the system is in a state of equilibrium. As such, it's reasonable to assume that the capacitor in the circuit is completely discharged, indicating a zero voltage across it. Similarly, no currents are flowing through the circuit elements as no potential difference exists to drive them. These constitute the initial conditions at the outset for the Runge-Kutta algorithm's iterations.

\begin{center}
    \Large No-Load Behavior and Single Reactive Element Analysis
\end{center}\\

Under a no-load condition, where the output is an open circuit, the system exhibits a distinct behavior. In such cases, it is evident that the system involves a single reactive element, specifically a capacitor. Consequently, the complexity of the system is reduced, and it can be described by a single ordinary differential equation (ODE). This ODE captures the relationship between the variables \(y\) and its derivative \(y'\), serving as a fundamental equation governing the system's behavior in the absence of a load. By analyzing and solving this ODE, we can gain insights into the dynamics and characteristics of the system when it operates without any external load connected to it.\\

Consequently, we can express the resulting equations for \(y\) and \(y'\) (the derivatives of \(y\)) as follows:\\

\begin{equation}
    y=[V_a]    
\end{equation}\\

\begin{equation}
    y'=[\frac{dV_a}{dt}]=[\frac{i_c}{C}]
\end{equation}\\

The expression for $y'$ of the Zener diode in the reverse-biased state can be obtained by utilizing Equation (1.65):

\begin{equation}
    \frac{d(e_2)}{dt}=\frac{1}{C}\sbracket{\bracket{\frac{\frac{e_1 - e_2}{2} - V_{th}}{R_b}}\cdot u\bracket{\frac{e_1 - e_2}{2} - V_{th}}-\frac{e_2-V_z}{R_s+R_z}} \hspace{0.1cm} \text{for} \hspace{0.1cm} e_2 \geq V_z
\end{equation}\\

\begin{equation}
    => y'=\frac{1}{C}[(u(\frac{V_2-e_2}{2}-V_{th})(\frac{\frac{V_2-e_2}{2}-V_{th}}{R_b})-\frac{V_a-V_z}{R_s+R_z}]
\end{equation}\\

The expression for $y'$ of the Zener diode in the cut-off state can be obtained by utilizing Equation (1.58):

\begin{equation}
     \frac{d(e_2)}{dt}=\frac{1}{C} \sbracket{\bracket{\frac{\frac{e_1 - e_2}{2} - V_{th}}{R_b}} \cdot u\bracket{\frac{e_1-e_2}{2}-V_{th}}} \hspace{0.5cm} \text{for} \hspace{0.1cm} e_2 < V_z
    \label{eq:de2dt_reverse_bias}
\end{equation}\\

\begin{equation}
    =>y'=\frac{1}{C}[u(\frac{V_2-V_a}{2}-V_t)\frac{\frac{V_2-V_a}{2}-V_t}{R_d}]
\end{equation}\\

The voltage of the capacitor for the next iteration in the Runge-Kutta algorithm can be calculated using Equations (2.7) and (2.9), taking into account the state of the Zener diode. Additional parameters can be determined by considering the input voltages $V_2$ and Va at each iteration.

To evaluate the state of the Zener diode, the voltage across it is examined at the beginning of each iteration. If this voltage is equal to or exceeds the threshold voltage $V_z$ of the Zener diode, it is considered to be in a reverse bias state. Conversely, if the voltage is below this threshold, the diode is in a cut-off state. The expression to calculate the voltage across the Zener diode at any given time can be derived from Equation (2.3).

\begin{equation}
    V_b=\begin{cases}
        V_a, V_b \leq V_z\\
        \frac{V_aR_z+V_zR_s}{R_s+R_z}, V_b>V_z
    \end{cases}
\end{equation}\\

The equation (2) can be utilized to determine the current flowing through the Zener diode:\\

\begin{equation}
    i_z=\frac{V_b-V_z}{R_z}u(V_b-V_z)
\end{equation}\\

In this scenario, the current flowing through the resistor Rs is equal to the current passing through the Zener diode. However, an alternative approach to determine this current through the resistor can also be derived using:\\

\begin{equation}
    i_s=\frac{V_a-V_b}{R_s}
\end{equation}\\

The Equation (1.9) provides the necessary information to determine the current through the diodes:\\

\begin{equation}
    i_d=u(\frac{V_2-V_a}{2}-V_t)(\frac{\frac{V_2-V_a}{2}-V_t}{R_d})
\end{equation}\\

The current through the capacitor can be expressed as follows:\\

\begin{equation}
    i_c=i_s-i_d
\end{equation}\\

\Large\textcolor{blue}{Resistive Load}\\

When considering a resistive load, the characteristics and behavior of the system can be further elaborated. Unlike reactive elements, such as capacitors or inductors, which introduce phase shifts and complex dynamics, resistive loads primarily consist of resistive elements like resistors. These resistive elements do not cause significant phase differences and primarily dissipate electrical energy as heat.

In the specific scenario of a resistive load, the system's equations and calculations revolve around the properties of resistive elements. These elements allow for simpler and more direct mathematical representations, enabling easier analysis and understanding of the system's behavior. By incorporating the appropriate equations and considering the characteristics of resistive loads, we can gain valuable insights into the system's performance, stability, and overall dynamics under these conditions.

In scenarios where it is evident that only one reactive element, specifically a capacitor, is present, the system can be effectively described by a single ordinary differential equation (ODE). Consequently, the resulting equations for $y$ and $y'$ can be expressed as follows:

\begin{equation}
    y=[V_a]
\end{equation}\\

\begin{equation}
    y'=[\frac{dV_a}{dt}]=[\frac{i_c}{C}]
\end{equation}\\

The equation (1.78) provides the derived expression for $y'$ of the Zener diode in its reverse-biased state.\\

\begin{equation}
    \frac{de_2}{dt}=\frac{1}{C}\left(u\left(\frac{V_2-e_2}{2}-V_{th}\right)\left(\frac{\frac{V_2-e_2}{2}-V_{th}}{R_b}\right)-\frac{e_2(R_L+R_z)-R_LV_Z}{R_zR_L+R_s(R_z+R_L)}\right), \text{if } \frac{R_Le_2}{R_s+R_L} \geq V_z
\end{equation}\\

\begin{equation}
    =>y'=\frac{1}{C}[u(\frac{V_2-V_a}{2}-V_t)(\frac{\frac{V_2-V_a}{2}}{R_d}-\frac{e_2(R_L+R_z)-R_LV_Z}{R_zR_L+R_s(R_z+R_L)})]
\end{equation}

The equation (1.65) yields the determined value for $y'$ of the Zener diode when it is in the cut-off state:\\

\begin{equation}
    \frac{d(e_2)}{dt}=\frac{1}{C}\sbracket{\bracket{\frac{\frac{e_1 - e_2}{2} - V_{th}}{R_b}}\cdot u\bracket{\frac{e_1 - e_2}{2} - V_{th}}-\frac{e_2-V_z}{R_s+R_z}} \hspace{0.1cm} \text{for} \hspace{0.1cm} e_2 \geq V_z
\end{equation}

\begin{equation}
    y'=\frac{1}{C}[u(\frac{V_2-V_a}{2}-V_t)(\frac{\frac{V_2-V_a}{2}-V_t}{R_d}-\frac{V_a}{(R_s+R_L)}]
\end{equation}\\

The equations (3.4.3) and (3.4.4) provide a means to calculate the voltage at the capacitor for the next iteration using the Runge-Kutta method, taking into account the state of the Zener diode. Additional parameters can be determined by utilizing the input voltages V2 and Va at each iteration.

The condition for the Zener modes remains the same as described previously. In this particular case, the expression for Vb can be derived using Equation (2.8.2).\\

\begin{equation}
    V_b=\begin{cases}
        \frac{\frac{V_a}{R_s}}{\frac{1}{R_s+\frac{1}{R_L}}}, V_b \leq V_z\\
        \frac{\frac{V_a}{R_s}+\frac{V_z}{R_z}}{\frac{1}{R_s}+\frac{1}{R_L}+\frac{1}{R_z}}, V_b>V_z
    \end{cases}=\begin{cases}
        \frac{R_LV_a}{R_s+R_L}, V_b \leq V_z\\
        \frac{R_zV_a+R_sV_z}{\frac{R_zR_s}{R_L}+R_z+R_s}, V_b>V_z
    \end{cases}
\end{equation}\\

The expression for the current flowing through the load resistance $R_L$ can be expressed as:\\

\begin{equation}
    I_{R_L}=\frac{V_b}{R_L}
\end{equation}\\

All other parameters can be determined using the previously derived equations.

\Large\textcolor{blue}{Inductive Load}\\

 The circuit under consideration involves two reactive elements, namely an inductor and a capacitor. This leads to a more complex system that can be described by a system of ordinary differential equations (ODEs). Consequently, the resulting equations for the vectors $y$ and $y'$, which represent the variables and their derivatives, can be formulated to capture the behavior of the circuit. Taking into account the inductive and capacitive elements, the analysis of this system requires an understanding of the dynamics, interactions, and interplay between these reactive components. By solving the system of ODEs, we can gain insights into the voltage and current relationships, energy transfers, and the overall behavior of the circuit under the influence of both the inductor and the capacitor.\\

In this scenario, the circuit comprises both an inductor and a capacitor as reactive elements. As a result, it is represented by a system of ODEs, where the equations involving the vectors $y$ and $y'$ describe its behavior. Solving these equations provides insights into the circuit's voltage-current relationships and energy dynamics.\\

\begin{equation}
    y=[\frac{V_a}{I_L}]
\end{equation}\\

\begin{equation}
    y'=[\begin{cases}
        \frac{dV_a}{dt}\\
        \frac{dI_L}{dt}
    \end{cases}]=[\begin{cases}
        \frac{i_c}{C}\\
        \frac{V_c}{L}
    \end{cases}]
\end{equation}\\

The equation (1.94) yields the derived expression for $y'$ of the Zener diode in its reverse-biased state:\\

\begin{equation}
    \begin{cases}
        \frac{di_L}{dt}=\frac{1}{L}(\frac{R_ze_2+R_sV_z-i_L(R_s(R_z+R_2)+R_2R_z)}{(R_z+R_s)})\\
        \frac{de_2}{dt}=\frac{1}{C}(u(\frac{V_2-e_2}{2}-V_{th})(\frac{V_2-e_2}{2}-V_{th})(\frac{\frac{V_2-e_2}{2}-V_{th}}{R_b})-\frac{e_2-V_z+i_LR_z}{R_z+R_s}),  \text{if } e_2-R_si_L \geq V_z
    \end{cases}
\end{equation}\\

\begin{equation}
    y'=[\begin{cases}
        \frac{1}{C}(u(\frac{V_2-V_a}{2}-V_t)(\frac{\frac{V_2-V_a}{2}-V_t}{R_d}-\frac{V_a-V_z+i_LR_z}{(R_z+R_s)})\\
        \frac{1}{L}(\frac{R_zV_a+R_sV_z-i_L(R_s(R_z+R_2)+R_2R_z)}{R_z+R_s})
    \end{cases}]
\end{equation}\\

The equation (1.89) provides the determined expression for $y'$ of the Zener diode when it is in the cut-off state.\\

\begin{equation}
    \begin{cases}
        \frac{di_L}{dt}=\frac{1}{L}(e_2-(R_s+R_2)i_L)\\
        \frac{de_2}{dt}=\frac{1}{C}\left(u\left(\frac{V_2-e_2}{2}-V_{th}\right)\left(\frac{V_2-e_2}{2}-V_{th}\right)\left(\frac{\frac{V_2-e_2}{2}-V_{th}}{R_b}\right)-i_L\right), \text{if } e_2-R_si_L<V_z
    \end{cases}
\end{equation}\\

\begin{equation}
    =>y'=[\begin{cases}
        \frac{1}{C}(u(\frac{V_2-V_a}{2}-V_t)(\frac{\frac{V_2-V_a}{2}-V_t}{R_d}-I_L)\\
        \frac{1}{L}(V_a-(R_s+R_2)I_L)
    \end{cases}]
\end{equation}\\

Equations $(3.5.3)$ and $(3.5.4)$ provide a means to determine the voltage across the capacitor and the current flowing through the inductor at the next iteration within the Runge-Kutta algorithm. These calculations depend on the state of the Zener diode. Other parameters can be obtained by considering the input voltage $V_2$, $V_a$, and $I_L$ at each iteration.

The conditions for the Zener modes remain consistent with what has been previously discussed. In this specific scenario, the value of $V_b$ can be derived from Equation $(2.14)$.\\


[need to change this equation to equation 3.5.5 in eugenes]
\begin{equation}
    V_b=\begin{cases}
        V_a-R_sI_L, V_b \leq V_z\\
        \frac{\frac{V_a}{R_s}-I_L+\frac{V_z}{R_z}}{\frac{1}{R_s+\frac{1}{R_z}}}, V_b>V_z
    \end{cases}=\begin{cases}
        V_a-R_sI_L, V_b \leq V_z\\
        \frac{R_z(V_a-I_LR_s)+V_zR_s}{R_s+R_z}, V_b>V_z
    \end{cases} 
\end{equation}\\

The expression for the current flowing through resistor R2 is as follows:\\

\begin{equation}
    I_{R_2}=I_L
\end{equation}\\

The voltage across the inductor Vc can be expressed as:\\

\begin{equation}
    V_c=V_b-R_2I_L
\end{equation}\\

The remaining parameters can be determined utilizing previously derived equations.\\

\Large\textcolor{blue}{Full Load}

The circuit being discussed involves two reactive elements, specifically an inductor and a capacitor. As a result, the behavior of the circuit can be accurately described by a system of ordinary differential equations (ODEs). These equations, involving the vectors $y$ and $y'$, capture the dynamics and interactions between the inductor and capacitor within the circuit.

Now, shifting our focus to the concept of full load, it is important to recognize that full load conditions can significantly impact the performance of a circuit or system. Under full load, the system operates at its maximum or rated capacity, experiencing the highest current or power demand. This state is crucial for evaluating the reliability, efficiency, and functionality of the circuit, ensuring that it can handle the load without compromising its performance.

When examining the behavior of a circuit under full load, considerations such as power consumption, heat dissipation, and voltage stability become paramount. Engineers and designers must carefully size and design the circuit to meet the demands of full load conditions, ensuring safe and efficient operation.

Understanding the behavior of a circuit with reactive elements, as described in your friend's message, allows us to evaluate its performance under full load accurately. By considering the derived equations for $y$ and $y'$ within the context of full load, we can assess the system's capability to handle maximum demands and determine its overall reliability and efficiency under these challenging operating conditions.

\begin{equation}
    y=[V_a\\
    I_L]
\end{equation}

\begin{equation}
    y'=[\frac{dV_a}{dt}\\
    \frac{dI_L}{dt}]=[\frac{i_c}{C}\\
    \frac{V_c}{L}]\\
\end{equation}


$y'$ for Zener in reverse-biased state has been found from the Equations (1.93, 1.94, 1.95, 1.96):\\

\begin{equation}
    y'=[\frac{1}{C}(u(\frac{V_2-e_2}{2}-V_{th})(\frac{V_2-e_2}{2}-V_{th})(\frac{\frac{V_2-e_2}{2}-V_{th}}{R_b})-\frac{V_a(R_z+R_L)+R_L(i_LR_z-V_z)}{(R_s(R_z+R_L)+R_LR_z)}\\
    \frac{1}{L}(R_L\frac{V_aR_z+(V_z-i_LR_z)R_s}{(R_s(R_z+R_L)+R_LR_z)}-R_2i_L)]
\end{equation}\\

The equation (2.21.6) provides the derived expression for $y'$ of the Zener diode when it is in the cut-off state.\\

\begin{equation}
    y'=[\frac{1}{C}(u((\frac{V_2-e_2}{2}-V_{th})(\frac{\frac{V_2-e_2}{2}-V_{th}}{R_b})-\frac{V_a+i_LR_L}{R_s+R_L}\\
    \frac{1}{L}(\frac{-i_L(R_s(R_L+R_2)+R_LR_2)+V_aR_L}{R_s+R_L})]
\end{equation}\\

By utilizing Equations (3.6.3) and (3.6.4), it is possible to determine the voltage across the capacitor and the current flowing through the inductor at the subsequent iteration in the Runge-Kutta algorithm. These calculations are contingent upon the state of the Zener diode. Moreover, by considering the input voltage V2, Va, and IL at each iteration, it becomes feasible to ascertain other parameters.\\

\begin{equation}
    V_b=\begin{cases}
        \frac{\frac{V_a}{R_s}-I_L}{\frac{1}{R_s}+\frac{1}{R_L}}, V_b \leq V_z\\
        \frac{\frac{V_a}{R_s}-I_L+\frac{V_z}{R_z}}{\frac{1}{R_s}+\frac{1}{R_L}+\frac{1}{R_z}}, V_b>V_z
    \end{cases}=\begin{cases}
        R_L\frac{V_a-I_LR_s}{R_L+R_s}, V_b \leq V_z\\
        \frac{R_zV_a-I_LR_sR_z+V_zR_s}{R_s+\frac{R_sR_z}{R_L}+R_z}, V_b>V_z
    \end{cases}
\end{equation}\\

The remaining parameters can be determined by employing previously derived equations.\\

\Large\textcolor{blue}{Switching in Motion}\\

In the realm of switching in motion, as your friend highlighted, the ability to connect or disconnect loads using switches while power is applied to the circuit is a crucial aspect. When dealing with resistive loads, the previously derived equations remain accurate, assuming an ideal resistor with no dynamic response to changes in voltage or current.

For connecting an inductive load, it becomes essential to ensure that the current through the inductor is maintained at $0A$. Under this condition, the aforementioned equations remain valid. However, the model described above does not encompass the scenario when an inductive load is disconnected. Previously, it was assumed that the switch is ideal, leading to instantaneous disconnection as the time interval ($\Delta t$) approaches zero ($\Delta t \to 0s$). However, in practical situations, when the switch is flipped, the current through the inductor will not instantaneously reduce to zero. As a result, the voltage across the inductor can be expressed using the equation [4]:

This highlights the need to consider the dynamics of inductive loads during switching in motion, as the instantaneous reduction of current to zero may not be achieved. Proper analysis and understanding of these dynamics are essential for accurate modeling and effective control of systems involving inductive loads.

    















\subsection{Numerical Algorithim}
\subsection{Timestep Scale}
\subsection{Maximial Diode Current}
\section{Simulation}
\subsection{Switches}
\subsection{Algorithmic Implementation}
MATLAB was used to implement the numerical analysis using RK4.
\begin{itemize}
	\item todo describe code operation get gpt-4 to do it
\end{itemize}
\lstinputlisting[caption=The MATLAB code used to derive the circuit parameters of the simple power supply]{code/powerSupply.m}

\pagebreak \lstinputlisting[caption=The MATLAB code used to derive the circuit parameters of the simple power supply]{code/powerSupply_noLoad.m}
\pagebreak \lstinputlisting[caption=The MATLAB code used to derive the circuit parameters of the simple power supply]{code/powerSupply_resistiveLoad.m}
\pagebreak \lstinputlisting[caption=The MATLAB code used to derive the circuit parameters of the simple power supply]{code/powerSupply_inductiveLoad.m}
\pagebreak \lstinputlisting[caption=The MATLAB code used to derive the circuit parameters of the simple power supply]{code/powerSupply_fullLoad.m}
\section{Visualisation}
\subsection{Overview}
\subsection{Usage Guide}
\subsection{Implementation}
\subsection{Switching Operation}

% S; Appendix
\section*{Appendix}
Add extra notes

Collaborative Approach for Enhanced Efficiency and Quality

In our project, we adopted a collaborative group approach, recognizing the benefits of collective brainstorming and problem-solving for each sub-problem. Rather than working individually and waiting for one team member to complete their section before moving on, we leveraged the power of teamwork to foster efficiency and produce higher-quality outcomes.

By working together, we harnessed the diverse perspectives, knowledge, and skills of each team member. This collaborative environment allowed us to generate a range of innovative solutions, benefiting from the collective wisdom and expertise of the group. Through active participation and open communication, we fostered a productive atmosphere where ideas were shared, refined, and built upon collaboratively.

It is important to emphasize that the work we have presented is a culmination of our joint efforts as a team. The report and video presentation are the result of our collaborative endeavors, where each member contributed their expertise and insights. By pooling our strengths and working collaboratively, we were able to achieve a higher level of quality and create a more comprehensive output than what could have been accomplished individually.

Our collaborative approach not only facilitated efficient problem-solving but also enhanced the overall quality of our work. By leveraging the collective intelligence of the team, we were able to explore different perspectives, validate ideas, and produce a well-rounded project that reflects the synergy of our collaborative efforts.

% S; Biblography
\bibliography{frontmatter/references}

\end{document}













